\documentclass[12pt,a4paper]{article}

\usepackage{geometry}
\geometry{
    left=2cm, 
    right=2cm,
    top=3cm,  
    bottom=2cm
}

\usepackage[english,spanish]{babel}
\usepackage[utf8]{inputenc}
\usepackage{amsmath}

\usepackage{graphicx}
\usepackage{wrapfig}

\usepackage{csquotes}
\usepackage{hyperref}
\usepackage[style=ieee]{biblatex}
\addbibresource{referencias.bib}

\usepackage{setspace}
\setstretch{1.5}
\setlength{\parindent}{0pt}

\begin{document}
\begin{titlepage}
    \begin{minipage}[c]{0.1\textwidth}
        \includegraphics[width=\textwidth]{logo_unam.jpg}
    \end{minipage}
    \begin{minipage}{0.8\textwidth}
        \centering
        {\Large\textbf{Universidad Nacional Autónoma de México}\\}
        {\large\textbf{Escuela Nacional de Estudios Superiores\\\underline{Unidad Morelia}}}
    \end{minipage}
    \begin{minipage}[c]{0.1\textwidth}
        \includegraphics[width=\textwidth]{logo_enes.jpg}
    \end{minipage}
    \vspace{3cm}

    \centering
    {\large{Ante Proyecto de:\\}}
    {\Large\textbf{Análisis Estadístico de Valores Nutricionales por Tipo de Dieta}}
    \vspace{2cm}

    {{PRESENTA:\\}}
    {\large\textbf{Alexis Uriel Aguilar Uribe}}
    \vspace{1cm} 

    {{PROFESORES:\\}}
    {\large\textbf{Dra.\ María Del Río Francos}}\\
    {\large\textbf{Dr.\ César Andrés Torres Miranda}}
    \vspace{2cm}

    {{GRADO\\}}
    {\large\textbf{Licenciatura en Tecnologías para la Información en Ciencias}}
    \vspace{2cm}

    \flushleft{
    {\textbf{Asignatura:\ }Estadística Descriptiva e Inferencial}
    \vspace{2cm}}

    \flushright{
    {\textbf{A:\ }\underline{21 de Mayo del 2025}}}
    \vfill
\end{titlepage}

\newpage

\tableofcontents

\newpage

\section{Presentación de los Datos}

    \subsection{Fuente de Datos}
    El conjunto de datos con el que se está trabajando para este trabajo 
    se encuentra en \cite{dataset_macronutrients}, publicado por la comunidad 
    de Kaggle. Los datos consisten de un conjunto de recetas de diferentes 
    dietas y cocinas, además incluye información de los macronutrientes de 
    cada receta.\\
    \cite{dataset_macronutrients} Aunque en la descripción ni en los metadatos del conjunto de datos se 
    haga mención de las fuentes explícitas de los datos ni el objetivo de 
    esta extracción, sí cuenta con una sección de cómo usar el conjunto de 
    datos, ideas de investigación y reconocimientos.\\
    De los apartados de cómo usar el conjunto de datos e ideas de investigación, 
    se encuentra una idea, implícita, de la información que se quería estudiar. 
    La principal información de interés se vuelve que es: el crear planes 
    alimenticios saludables, ya sea usando las recetas proporcionadas o creando 
    unas nuevas basadas en una dieta y cocina, y el estudiar la relación entre 
    dieta y salud.\\
    Del apartado de reconocimientos, se concluye que las recetas fueron 
    proporcionadas por diferentes creadores de las mismas y demás contribuidores 
    al conjunto de datos. 

    \subsection{Interés del Estudio}
    Se consultó \cite{marvastipopular} en sus 
    capítulos 4 y 8, de donde se proporciona un mejor entendimiento de la 
    importancia de los macronutrientes y una descripción general de las 
    dietas en este trabajo, resultando interesante que en cada dieta se 
    consumen diferentes alimentos y productos con ciertas características 
    para ya sea respetar alguna creencia, fundamento o cota de macronutrientes. 
    De esto último, proporciona un indicio de que existe una diferencia entre 
    las dietas a nivel de sus aportes nutricionales, por lo tanto, lo que se 
    quiere realizar es probar esta diferencia de manera significativa haciendo 
    uso de la estadística.

    \subsection{Variables del Conjunto de Datos}
    El conjunto de datos consta de las siguientes variables. Se menciona su 
    nombre, el tipo de variable y sus valores (en total y únicos):
    \begin{center}
        \begin{tabular}{|c|c|c|c|c|}
            \hline
            Variable & Nombre & Tipo & Cantidad de Datos & Valores Únicos\\
            \hline
            1 & Diet\_type & Cualitativa Nominal & 7806 & 5 \\
            2 & Recipe\_name & Cualitativa Nominal & 7806 & 7062\\
            3 & Cuisine\_type & Cualitativa Nominal & 7806 & 19\\
            4 & Protein(g) & Cuantitativa Continua & 7806 & 6060\\
            5 & Carbs(g) & Cuantitativa Continua & 7806 & 6618\\
            6 & Fat(g) & Cuantitativa Continua & 7806 & 6322\\
            \hline
        \end{tabular}
    \end{center}

\newpage

\section{Estadística Descriptiva}

\newpage

\section{Muestreo e Intervalos de Confianza}

\newpage

\end{document}