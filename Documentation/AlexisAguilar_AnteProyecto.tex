\documentclass[12pt,a4paper]{article}

\usepackage{geometry}
\geometry{
    left=2cm, 
    right=2cm,
    top=3cm,  
    bottom=2cm
}

\usepackage[english,spanish]{babel}
\usepackage[utf8]{inputenc}
\usepackage{amsmath}

\usepackage{graphicx}
\usepackage{wrapfig}

\usepackage{csquotes}
\usepackage{hyperref}
\usepackage[style=ieee]{biblatex}
\addbibresource{referencias.bib}

\usepackage{setspace}
\setstretch{1.5}
\setlength{\parindent}{0pt}

\usepackage{enumitem}

\usepackage{xcolor}

\begin{document}
\begin{titlepage}
    \begin{minipage}[c]{0.1\textwidth}
        \includegraphics[width=\textwidth]{Resources/logo_unam.jpg}
    \end{minipage}
    \begin{minipage}{0.8\textwidth}
        \centering
        {\Large\textbf{Universidad Nacional Autónoma de México}\\}
        {\large\textbf{Escuela Nacional de Estudios Superiores\\\underline{Unidad Morelia}}}
    \end{minipage}
    \begin{minipage}[c]{0.1\textwidth}
        \includegraphics[width=\textwidth]{Resources/logo_enes.jpg}
    \end{minipage}
    \vspace{3cm}

    \centering
    {\large{Ante Proyecto de:\\}}
    {\Large\textbf{Análisis Estadístico de Valores Nutricionales por Tipo de Dieta}}
    \vspace{2cm}

    {{PRESENTA:\\}}
    {\large\textbf{Alexis Uriel Aguilar Uribe}}
    \vspace{1cm} 

    {{PROFESORES:\\}}
    {\large\textbf{Dra.\ María Del Río Francos}}\\
    {\large\textbf{Dr.\ César Andrés Torres Miranda}}
    \vspace{2cm}

    {{GRADO\\}}
    {\large\textbf{Licenciatura en Tecnologías para la Información en Ciencias}}
    \vspace{2cm}

    \flushleft{
    {\textbf{Asignatura:\ }Estadística Descriptiva e Inferencial}
    \vspace{2cm}}

    \flushright{
    {\textbf{A:\ }\underline{21 de Mayo del 2025}}}
    \vfill
\end{titlepage}

\newpage

\tableofcontents

\newpage

\section{Presentación de los Datos}

    \subsection{Fuente de Datos}
    El conjunto de datos con el que se está trabajando para este trabajo 
    se encuentra en \cite{dataset_macronutrients}, publicado por la comunidad 
    de Kaggle. Los datos consisten de un conjunto de recetas de diferentes 
    dietas y cocinas, además incluye información de los macronutrientes que 
    aporta cada receta.\\
    \cite{dataset_macronutrients} Aunque en la descripción ni en los metadatos del conjunto de datos se 
    haga mención de las fuentes explícitas de los datos ni el objetivo de 
    esta extracción, sí cuenta con una sección de cómo usar el conjunto de 
    datos, ideas de investigación y reconocimientos.\\
    De los apartados de cómo usar el conjunto de datos e ideas de investigación, 
    se encuentra una idea, implícita, de la información que se quería estudiar. 
    La principal información de interés se vuelve que es: el crear planes 
    alimenticios saludables, ya sea usando las recetas proporcionadas o creando 
    unas nuevas basadas en una dieta y cocina, y el estudiar la relación entre 
    dieta y salud.\\
    Del apartado de reconocimientos, se concluye que las recetas fueron 
    proporcionadas por diferentes creadores de las mismas y demás contribuidores 
    al conjunto de datos. 

    \subsection{Interés del Estudio}
    Se consultó \cite{marvastipopular} en sus 
    capítulos 4 y 8, de donde se proporciona un mejor entendimiento de la 
    importancia de los macronutrientes y una descripción general de las 
    dietas en este trabajo, resultando interesante que en cada dieta se 
    consumen diferentes alimentos y productos con ciertas características 
    para ya sea respetar alguna creencia, fundamento o cota de macronutrientes. 
    De esto último, proporciona un indicio de que existe una diferencia entre 
    las dietas a nivel de sus aportes nutricionales, por lo tanto, lo que se 
    quiere realizar es probar esta diferencia de manera significativa haciendo 
    uso de la estadística.

    \subsection{Variables del Conjunto de Datos}
    El conjunto de datos consta de las siguientes variables. Se menciona su 
    nombre, el tipo de variable y sus valores (en total y únicos):
    \begin{center}
        \begin{tabular}{|c|c|c|c|c|}
            \hline
            Variable & Nombre & Tipo & Cantidad de Datos & Valores Únicos\\
            \hline
            1 & Diet\_type & Cualitativa Nominal & 7806 & 5 \\
            2 & Recipe\_name & Cualitativa Nominal & 7806 & 7062\\
            3 & Cuisine\_type & Cualitativa Nominal & 7806 & 19\\
            4 & Protein(g) & Cuantitativa Continua & 7806 & 6060\\
            5 & Carbs(g) & Cuantitativa Continua & 7806 & 6618\\
            6 & Fat(g) & Cuantitativa Continua & 7806 & 6322\\
            \hline
        \end{tabular}
    \end{center}
    La variable Recipe\_Name no es relevante para este trabajo pero figura 
    dentro del dataset. Se hace mención que el conjunto de datos no presenta 
    valores faltantes.

\newpage

\section{Estadística Descriptiva}
    Debido a que cada receta puede aportar una amplia variedad de valores 
    en sus macronutrientes, esto podría dificultar la comparación entre 
    los aportes nutricionales de las dietas. Por ello, para reducir este 
    impacto de sesgo, se aplico una normalización a los valores, es decir, 
    los macronutrientes de cada receta se dividió por el total de macronutrientes 
    que aportaba la receta, para así manejar los aportes proporcionales de 
    cada macronutriente en cada una de las recetas. De aquí en adelante, cuando 
    se mencionan los aportes nutricionales o por macronutriente de una receta, 
    siempre se hace referencia a estos aportes proporcionales respecto al total de 
    macronutrientes en una receta. 

    \subsection{Descripción de los Valores de las Variables}
    Para el presente trabajo se harán uso de las siguientes variables, se 
    acompañan con una descripción de su significado:
    \begin{itemize}[label=\textbullet]
        \item \textbf{Diet\_type}: Variable que representa el tipo de 
        dieta (DASH, keto, mediterránea, paleo, vegana) a la que 
        pertenece una receta. Con esta variable se va permitir estratificar 
        las recetas y estudiarlas de una manera más granular, es decir, 
        por tipo de dieta para llegar a conformar hipótesis sobre lo qué 
        está pasando en una dieta o entre las diferentes dietas.
        \item \textbf{Cuisine\_type}: Variable que representa a qué (estilo 
        de) cocina o región (mexicana, americana, italiana) pertenece una 
        receta. Al usarla va a permitir el comparar cómo son las recetas 
        de una dieta en comparación con otras regiones, en específico el 
        como se compara la dieta mediterránea en el mediterráneo en comparación 
        con otras regiones geográficas.
        \item \textbf{Protein(g)}: Después de la transformación, representa el 
        porcentaje, respecto al total de macronutrientes, de proteínas que son 
        aportados por una receta. El usar las proteínas va a permitir la 
        comparación entre diferentes dietas, siendo esto el eje central del trabajo
        \item \textbf{Carbs(g)}: Después de la transformación, representa el 
        porcentaje, respecto al total de macronutrientes, de carbohidratos que 
        son aportados por una receta. Siendo otro de los macronutrientes de una 
        comida, se vuelve relevante para la comparación entre recetas y dietas.
        \item \textbf{Fat(g)}: Después de la transformación, representa el 
        porcentaje, respecto al total de macronutrientes, de grasas que son 
        aportados por una receta. Y el último macronutriente, como en los 
        anteriores, se vuelve una variable relevante para la comparación entre dietas.
    \end{itemize}

    \subsection{Medidas de Tendencia Central y Dispersión}
    Realizando el resumen de las medidas, se tiene:
    \begin{center}
        \begin{tabular}{|c|ccc|}
            \hline
            Medida & Carbs(g) & Protein(g) & Fat(g) \\
            \hline
            Media               & 0.433471 & 0.234762 & 0.331767 \\
            $Q_1$               & 0.205251 & 0.110188 & 0.184583 \\
            $Q_2$               & 0.432028 & 0.190931 & 0.314359 \\
            $Q_3$               & 0.635058 & 0.338059 & 0.464532 \\
            Desviación Estándar & 0.256032 & 0.163886 & 0.194920 \\
            Mínimo              & 0.000330 & 0.000000 & 0.000000 \\
            Máximo              & 1.000000 & 0.887557 & 0.997940 \\
            Asimetría de Fisher & 0.189556 & 0.922401 & 0.461455 \\
            \hline
        \end{tabular}
    \end{center}
    Debido a que son medidas sobre todos los datos, sin estratificar, 
    se tiene que no hay una referencia de lo que se espera obtener y 
    parte de la información que contienen queda diluida o desvanecida. 
    Esto debido a que las dietas como la vegana es baja en proteínas y 
    la keto en carbohidratos \cite{marvastipopular}, por lo que cualquier 
    suposición no se podría sostener sobre todos las dietas.\\

    Aún así, se reportan bajos valores en proteínas en comparación 
    con los carbohidratos y grasas si se hace uso de la mediana ($Q_2$), 
    dicho así: el cincuenta por ciento de las recetas tienen a lo mucho  
    $19.09\%$ de proteínas, en comparación con el $43.20\%$ de carbohidratos 
    y el $31.43\%$ de grasas. Esto es un indicio de que las recetas, en general, 
    tienden a ser altas en carbohidratos y grasas entre las diferentes dietas y 
    cocinas; mientras que son bajas en proteínas.\\
    Este último punto puede ser apoyado si se considera la media de los 
    macronutrientes, que siguen este prototipo de aportes dominantes de 
    carbohidratos.\\

    Si se gráfica la distribución de los macronutrientes se tiene que, debido 
    a la asimetría y a la desviación estándar, contienen datos atípicos en proteínas y grasas en una región 
    positiva respecto a la mediana, y esto se relaciona con lo mencionado de que 
    una receta no tiende a un aporte alto de proteínas. Y si se consider el rango 
    intercuartil, se observa que en estos macronutrientes es menor, en comparación, 
    que con el de los carbohidratos, esto muestra la diversidad en las recetas respecto 
    a los posibles valores que puedan tomar para sus aportes nutricionales.
    \begin{center}
        \includegraphics[width=0.75\textwidth]{Resources/2_02_plot_01.png}
    \end{center}

    Debido a que existe la presencia de datos atípicos, lo más adecuado es tratarlos 
    de manera estratificada, por tipo de dieta. Esto debido a que tratarlos de manera 
    general podría evocar que ciertas dietas queden menos representadas en comparación 
    con otras o que incluso se pierda información para consecuentes procesos. Y al 
    tratar los valores atípicos dentro de cada dieta permite reducir el impacto de 
    perder información valiosa y se siga conservando las recetas relevantes para una dieta.

    \subsection{Estratificación de Valores Cuantitativos}
    La variable Diet\_type es la principal que se emplea para la 
    estratificación de las recetas, debido a que permite seprarlas 
    según una criterio bien definida, a qué dieta pertenecen. Para cada 
    una de las cinco dietas se presentan los datos tabulados de sus 
    medidas de tendencia central y dispersión junto con su histograma 
    de los valores en sus macronutrientes.\\

    \subsubsection{Dieta DASH}
        Una receta de esta dieta tendrá que, en promedio, el $55\%$ de sus 
        macronutrientes sons carbohidratos (provenientes de frutas, vegetales 
        y granos enteros); el $25\%$ son grasas que, por su naturaleza, son 
        saludables; y el $20\%$ son proteínas, las cuáles provienen de carnes margas.
        \begin{center}
            \begin{tabular}{|c|ccc|}
                \hline
                Medida & Carbs(g) & Protein(g) & Fat(g) \\
                \hline
                Media               & 0.549425 & 0.196241 & 0.254334  \\
                $Q_1$               & 0.331143 & 0.068931 & 0.103381  \\
                $Q_2$               & 0.555219 & 0.156626 & 0.234742  \\
                $Q_3$               & 0.757917 & 0.282629 &	0.371292  \\
                Desviación Estándar & 0.278850 & 0.162871 & 0.194078  \\
                Mínimo              & 0.001526 & 0.000000 & 0.000000  \\
                Máximo              & 1.000000 & 0.833467 & 0.973404  \\
                Asimetría de Fisher & -0.057984 & 1.101171 & 0.732534  \\
                \hline
            \end{tabular}
        \end{center}
        Debido a que existe un sesgo positivo notable en las contribuciones de 
        proteínas y grasas, se tiene que las recetas van a tender a tener bajos 
        aportes de estos macronutrientes y que si tienen un alto aporte se 
        consideraría una receta atípica dentro de la dieta.
        \begin{center}
            \includegraphics[width=0.75\textwidth]{Resources/2_03_plot_01.png}
        \end{center}

    \subsubsection{Dieta Keto}
        Una receta de esta dieta tendrá que, en promedio, el $50\%$ de 
        sus macronutrientes son grasas, esto se relaciona con el hecho de 
        que se intenta inducir la ketosis (principio en que se basa esta         
        dieta); el $30\%$ son proteínas, notando que se intenta reducir 
        el consumo de carbohidratos; y el $20\%$ son carbohidratos, 
        resaltando ser una dieta baja en carbohidratos.
        \begin{center}
            \begin{tabular}{|c|ccc|}
                \hline
                Medida & Carbs(g) & Protein(g) & Fat(g) \\
                \hline
                Media               & 0.200879 & 0.301777 & 0.497344  \\
                $Q_1$               & 0.085517 & 0.158284 & 0.405354  \\
                $Q_2$               & 0.157348 & 0.302900 & 0.505751  \\
                $Q_3$               & 0.267535 & 0.409453 & 0.591887  \\
                Desviación Estándar & 0.160609 & 0.167027 & 0.166572  \\
                Mínimo              & 0.002060 & 0.000000 & 0.000000  \\
                Máximo              & 1.000000 & 0.856868 & 0.997940  \\
                Asimetría de Fisher & 1.634945 & 0.314795 & -0.147406 \\
                \hline
            \end{tabular}
        \end{center}
        Como la proporciones de carbohidratos cuenta con un sesgo positivo, se 
        tiene que refuerza el hecho de ser una dieta baja en carbohidratos. 
        De los aportes de grasas, se observa que su sesgo es despreciable implicando 
        que existen recetas tanto con aportes altos de este macronutriente (lo que se 
        busca) mientras que hay recetas con una contribución baja o nula del mismo.
        \begin{center}
            \includegraphics[width=0.75\textwidth]{Resources/2_03_plot_02.png}
        \end{center}

    \subsubsection{Dieta Mediterránea}
        Una receta de esta dieta tendrá que, en promedio, el $42\%$ de sus 
        macronutrientes son carbohidratos, esto debido a un alto consumo de 
        productos como, frutas, vegetales y granos enteros; el $30\%$ son 
        grasas, resaltando un alto consumo de nueces y aceite de oliva, como 
        también un consumo moderado de pescado; y el $28\%$ son proteínas, 
        vinculado con un consumo moderado de pescado y aves de corral, y un 
        bajo consumo de carnes rojas. 
        \begin{center}
            \begin{tabular}{|c|ccc|}
                \hline
                Medida & Carbs(g) & Protein(g) & Fat(g) \\
                \hline
                Media               & 0.424493 & 0.279357 & 0.296150  \\
                $Q_1$               & 0.249955 & 0.159633 & 0.180357  \\
                $Q_2$               & 0.439382 & 0.227883 & 0.268336  \\
                $Q_3$               & 0.607531 & 0.377820 & 0.390404  \\
                Desviación Estándar & 0.214325 & 0.162853 & 0.160783  \\
                Mínimo              & 0.006733 & 0.005036 & 0.001731  \\
                Máximo              & 0.992746 & 0.887557 & 0.968722  \\
                Asimetría de Fisher & -0.096055 & 0.955922 & 0.869493 \\
                \hline
            \end{tabular}
        \end{center}
        La proporción de proteínas está segada positivamente y junto con una 
        alta acumulación de recetas con bajo consumo de proteínas, se tiene que 
        esta dieta figura como una con bajo consumo de alimentos ricos en proteínas. 
        \begin{center}
            \includegraphics[width=0.75\textwidth]{Resources/2_03_plot_03.png}
        \end{center}

    \subsubsection{Dieta Paleo}
        Una receta de esta dieta tendrá que, en promedio, el $38\%$ de sus 
        macronutrientes son grasas y el $37\%$ son carbohidratos, esto se 
        relaciona con el consumo de productos como frutas, vegetales, nueces 
        y semillas; y el $25\%$ son proteínas cuyas principales fuentes son 
        carnes margas y pescado.
        \begin{center}
            \begin{tabular}{|c|ccc|}
                \hline
                Medida & Carbs(g) & Protein(g) & Fat(g) \\
                \hline
                Media               & 0.371307 & 0.249693 & 0.379000  \\
                $Q_1$               & 0.192399 & 0.102963 & 0.256579  \\
                $Q_2$               & 0.351300 & 0.205532 & 0.382447  \\
                $Q_3$               & 0.515054 & 0.375392 & 0.488116  \\
                Desviación Estándar & 0.221506 & 0.175031 & 0.175471  \\
                Mínimo              & 0.003612 & 0.000000 & 0.001404  \\
                Máximo              & 0.987368 & 0.858503 & 0.968835  \\
                Asimetría de Fisher & 0.488656 & 0.711408 & 0.312673  \\
                \hline
            \end{tabular}
        \end{center}
        Se observa como las recetas tienden a tener una contribución moderada de 
        carbohidratos y grasas, esto se relaciona con los principales alimentos 
        que son consumidos en esta dieta. Mientras que sus aportes de proteínas 
        son bajas en comparación con los otros dos macronutrientes.
        \begin{center}
            \includegraphics[width=0.75\textwidth]{Resources/2_03_plot_04.png}
        \end{center}

    \subsubsection{Dieta Vegana}
        Una receta de esta dieta tendrá que, en promedio, el $60\%$ de 
        sus macronutrientes son carbohidratos, que provienen de fuentes 
        como vegetales, frutas, cereales y legumbres; el $25\%$ son grasas, 
        relacionadas con el consumo de nueces y semillas; y el $15\%$ son 
        proteínas, esto debido a un nulo consumo de alimentos de origen 
        animal y que estas fuentes son reemplazadas por fuentes vegetales.
        \begin{center}
            \begin{tabular}{|c|ccc|}
                \hline
                Medida & Carbs(g) & Protein(g) & Fat(g) \\
                \hline
                Media               & 0.593968 & 0.148489 & 0.257543  \\
                $Q_1$               & 0.504070 & 0.085339 & 0.142575  \\
                $Q_2$               & 0.626246 & 0.139688 & 0.231518  \\
                $Q_3$               & 0.714679 & 0.190381 & 0.344529  \\
                Desviación Estándar & 0.171203 & 0.086088 & 0.160277  \\
                Mínimo              & 0.000330 & 0.001921 & 0.000112  \\
                Máximo              & 0.986872 & 0.647416 & 0.994887  \\
                Asimetría de Fisher & 0.189556 & 0.922401 & 0.461455  \\
                \hline
            \end{tabular}
        \end{center}
        De las proporciones de proteínas, se resalta una alta acumulación de 
        recetas con bajos aportes de proteínas, esto hace de esta dieta una 
        con bajo consumo de proteínas. Lo último debido a que las principales 
        fuentes proteínas son animales y haciendo que los aportes de carbohidratos 
        sean altos en comparación con los otros dos macronutrientes.
        \begin{center}
            \includegraphics[width=0.75\textwidth]{Resources/2_03_plot_05.png}
        \end{center}

    \subsubsection{Gráfico de Cajas y Bigotes de la distribución de Macronutrientes por Dieta}
        Se anexan las gráficas de cajas y bigotes de las distribuciones de los 
        macronutrientes por dieta para apoyar las observaciones realizadas anteriormente.
        \begin{center}
            \includegraphics[width=0.75\textwidth]{Resources/2_03_plot_06.png}
        \end{center}

\newpage

\section{Muestreo e Intervalos de Confianza}
    Como las tres variables cuantitativas tienen el mismo nivel 
    de relevancia, se opta por usar los Carbs(g) como atributo 
    para el Muestreo. Y para ambos muestreos se realizan de 
    tamaño $50$ y, usando la Regla de Sturges, se emplean 7 clases 
    o bins.
    
    \subsection{Muestreo Aleatorio Simple}
        
    \subsubsection{Resultados del Muestreo}
        \begin{center}
            \begin{tabular}{ccccc}
                0.17269271 & 0.46650541 & 0.75534765 & 0.47087379 & 0.63277125 \\
                0.07006336 & 0.05484247 & 0.65481006 & 0.16118891 & 0.2385986  \\
                0.4684997  & 0.24332505 & 0.62191414 & 0.34102783 & 0.62236612 \\
                0.39206706 & 0.08786462 & 0.37490208 & 0.03025985 & 0.46368715 \\
                0.50368913 & 0.25911022 & 0.64134393 & 0.62119952 & 0.67296013 \\
                0.69099174 & 0.5093633  & 0.3897762  & 0.50870391 & 0.62610896 \\
                0.16637611 & 0.00205997 & 0.83795162 & 0.70500232 & 0.39452352 \\
                0.96228571 & 0.3650539  & 0.73331589 & 0.2995115  & 0.17188455 \\
                0.7579386  & 0.35745213 & 0.5351373  & 0.68094919 & 0.19350238 \\
                0.32026529 & 0.62170193 & 0.13598024 & 0.96879183 & 0.30912958 \\
            \end{tabular}
        \end{center}

    \subsubsection{Z-Scores del Muestreo}
        Se tiene que la media muestral $\overline{x}$ es $0.445313$ y la 
        desviación estándar muestral $s$ es $0.246293$.
        \begin{center}
            \begin{tabular}{ccccc}
                -1.10689678 &  0.08604412 &  1.25880389 &  0.10378063 &  0.76111813 \\
                -1.52359336 & -1.58539332 &  0.85060029 & -1.15360459 & -0.83930511 \\
                 0.09414135 & -0.82011471 &  0.717036   & -0.42342106 &  0.71887111 \\
                -0.21619111 & -1.45131653 & -0.28588451 & -1.68520391 &  0.07460139 \\
                 0.23701778 & -0.75602365 &  0.79592498 &  0.71413449 &  0.92429335 \\
                 0.99750546 &  0.26005608 & -0.22549249 &  0.25737883 &  0.73406782 \\
                -1.13254349 & -1.79970131 &  1.59419324 &  1.05439133 & -0.20621739 \\
                 2.09901562 & -0.32587018 &  1.16935033 & -0.59198603 & -1.11017805 \\
                 1.2693237  & -0.35673498 &  0.36470391 &  0.95673061 & -1.02240518 \\
                -0.50772129 &  0.71617437 & -1.25595705 &  2.12543183 & -0.55293458 \\
            \end{tabular}
        \end{center}

    \subsubsection{Tabla de Frecuencias}
        \begin{center}
            \begin{tabular}{|c|c|c|c|c|}
                \hline
                Marca de Clase & Puntaje Z & Frecuencia Absoluta & Frecuencia Relativa & Frecuencia Acumulada \\
                \hline
                0.071112 & -1.519335 & 6.000000 & 0.120000 & 0.120000 \\
                0.209217 & -0.958601 & 8.000000 & 0.160000 & 0.280000 \\
                0.347321 & -0.397868 & 10.000000 & 0.200000 & 0.480000 \\
                0.485426 &  0.162865 & 8.000000 & 0.160000 & 0.640000 \\
                0.623530 &  0.723599 & 11.000000 & 0.220000 & 0.860000 \\
                0.761635 &  1.284332 & 4.000000 & 0.080000 & 0.940000 \\
                0.899740 &  1.845065 & 3.000000 & 0.060000 & 1.000000 \\
                \hline
            \end{tabular}
        \end{center} 

    \subsubsection{Medidas de Tendencia Central y Dispersión Muestrales}
        \begin{center}
            \begin{tabular}{|c|c|}
                \hline
                Medida muestral & Carbs(g) \\
                \hline
                Media & 0.445313 \\
                $Q_1$ & 0.247271 \\
                $Q_2$ & 0.465096 \\
                $Q_3$ & 0.631106 \\
                Desviación Estándar & 0.246293 \\
                Mínimo & 0.002060 \\
                Máximo & 0.968792 \\
                Asimetría de Fisher & 0.077583 \\
                \hline
                \end{tabular}
        \end{center}
    
    \subsubsection{Comparativa con las Medidas Poblacionales}
        En algunas métricas como lo son la media, el primer y segundo cuartil, 
        la desviación estándar, el máximo y la asimetría muestrales 
        difieren de sus respectivos valores poblacionales. De los valores, era 
        esperado que todos difieran aunque fuera un poco, salvo la media; para 
        probar esto último sería necesario de aplicar una prueba de hipótesis.\\
        Si se compara con la distribución poblacional, se tiene que la 
        distribución se conserva hasta cierto punto.
        \begin{center}
            \includegraphics[width=0.75\textwidth]{Resources/3_01_plot_01.png}
        \end{center}
    
    \subsection{Muestreo Aleatorio Estratificado}

    \subsubsection{Resultados del Muestreo}
        \begin{center}
            \begin{tabular}{ccccc}
                \textcolor{magenta}{0.44793647} & \textcolor{magenta}{0.18250769} & \textcolor{magenta}{0.63935358} & \textcolor{magenta}{0.21435184} & \textcolor{magenta}{0.91825902} \\
                \textcolor{magenta}{0.70068017} & \textcolor{magenta}{0.55288489} & \textcolor{magenta}{0.49903342} & \textcolor{magenta}{0.74194743} & \textcolor{magenta}{0.39776892} \\
                \textcolor{magenta}{0.68474824} & \textcolor{orange}{0.21756679} & \textcolor{orange}{0.64524101} & \textcolor{orange}{0.11898573} & \textcolor{orange}{0.11534155} \\
                \textcolor{orange}{0.51769646} & \textcolor{orange}{0.84650821} & \textcolor{orange}{0.08359541} & \textcolor{orange}{0.24333481} & \textcolor{orange}{0.42687973} \\
                \textcolor{orange}{0.04969789} & \textcolor{cyan}{0.6525685}  & \textcolor{cyan}{0.09963134} & \textcolor{cyan}{0.13678711} & \textcolor{cyan}{0.01271082} \\
                \textcolor{cyan}{0.60119711} & \textcolor{cyan}{0.53384367} & \textcolor{cyan}{0.71907757} & \textcolor{cyan}{0.58809768} & \textcolor{cyan}{0.37297297} \\
                \textcolor{cyan}{0.61700427} & \textcolor{cyan}{0.75929204} & \textcolor{purple}{0.34345027} & \textcolor{purple}{0.15223487} & \textcolor{purple}{0.57504277} \\
                \textcolor{purple}{0.69714352} & \textcolor{purple}{0.32857582} & \textcolor{purple}{0.39527363} & \textcolor{purple}{0.19990062} & \textcolor{purple}{0.30321616} \\
                \textcolor{teal}{0.3849578}  & \textcolor{teal}{0.65746746} & \textcolor{teal}{0.69443097} & \textcolor{teal}{0.65933286} & \textcolor{teal}{0.42047204} \\
                \textcolor{teal}{0.37633321} & \textcolor{teal}{0.7136477}  & \textcolor{teal}{0.53091451} & \textcolor{teal}{0.63226859} & \textcolor{teal}{0.50493571} 
            \end{tabular}
        \end{center}
        Leyenda de colores: \textcolor{magenta}{DASH}, \textcolor{orange}{Keto}, \textcolor{cyan}{Mediterránea}, 
        \textcolor{purple}{Paleo} y \textcolor{teal}{Vegana}.

    \subsubsection{Z-Scores del Muestreo}
        Se tiene que la media muestral $\overline{x}$ es $0.458142$ y la 
        desviación estándar muestral $s$ es $0.233125$.
        \begin{center}
            \begin{tabular}{ccccc}
                \textcolor{magenta}{-0.04377718} & \textcolor{magenta}{-1.18234679} & \textcolor{magenta}{ 0.77731576} & \textcolor{magenta}{-1.04574978} & \textcolor{magenta}{ 1.97369417} \\
                \textcolor{magenta}{ 1.04037915} & \textcolor{magenta}{ 0.40640414} & \textcolor{magenta}{ 0.17540563} & \textcolor{magenta}{ 1.21739704} & \textcolor{magenta}{-0.2589733}  \\
                \textcolor{magenta}{ 0.97203835} & \textcolor{orange}{-1.03195909} & \textcolor{orange}{ 0.80257021} & \textcolor{orange}{-1.45482732} & \textcolor{orange}{-1.47045922} \\
                \textcolor{orange}{ 0.25546167} & \textcolor{orange}{ 1.66591559} & \textcolor{orange}{-1.60663578} & \textcolor{orange}{-0.92142592} & \textcolor{orange}{-0.13410111} \\
                \textcolor{orange}{-1.75204084} & \textcolor{cyan}{ 0.83400182} & \textcolor{cyan}{-1.5378489}  & \textcolor{cyan}{-1.37846742} & \textcolor{cyan}{-1.91069868} \\
                \textcolor{cyan}{ 0.61364176} & \textcolor{cyan}{ 0.32472592} & \textcolor{cyan}{ 1.11929568} & \textcolor{cyan}{ 0.55745111} & \textcolor{cyan}{-0.36533674} \\
                \textcolor{cyan}{ 0.68144734} & \textcolor{cyan}{ 1.29179758} & \textcolor{purple}{-0.49197579} & \textcolor{purple}{-1.3122035}  & \textcolor{purple}{ 0.50145144} \\
                \textcolor{purple}{ 1.0252085}  & \textcolor{purple}{-0.55578048} & \textcolor{purple}{-0.26967698} & \textcolor{purple}{-1.10773898} & \textcolor{purple}{-0.66456198} \\
                \textcolor{teal}{-0.31392726} & \textcolor{teal}{ 0.85501613} & \textcolor{teal}{ 1.0135729}  & \textcolor{teal}{ 0.86301786} & \textcolor{teal}{-0.16158718} \\
                \textcolor{teal}{-0.35092284} & \textcolor{teal}{ 1.096004}   & \textcolor{teal}{ 0.31216114} & \textcolor{teal}{ 0.74692438} & \textcolor{teal}{ 0.20072383} \\
            \end{tabular}
        \end{center}
        Leyenda de colores: \textcolor{magenta}{DASH}, \textcolor{orange}{Keto}, \textcolor{cyan}{Mediterránea}, 
        \textcolor{purple}{Paleo} y \textcolor{teal}{Vegana}.

    \subsubsection{Tabla de Frecuencias}
        \begin{center}
            \begin{tabular}{|c|c|c|c|c|}
                \hline
                Marca de Clase & Puntaje Z & Frecuencia Absoluta & Frecuencia Relativa & Frecuencia Acumulada \\
                \hline
                0.077393 & -1.633242 & 7.000000 & 0.140000 & 0.140000 \\
                0.206757 & -1.078329 & 6.000000 & 0.120000 & 0.260000 \\
                0.336121 & -0.523416 & 8.000000 & 0.160000 & 0.420000 \\
                0.465485 & 0.031498 & 6.000000 & 0.120000 & 0.540000 \\
                0.594849 & 0.586411 & 13.000000 & 0.260000 & 0.800000 \\
                0.724213 & 1.141324 & 8.000000 & 0.160000 & 0.960000 \\
                0.853577 & 1.696238 & 2.000000 & 0.040000 & 1.000000 \\
                \hline
            \end{tabular}
        \end{center}

    \subsubsection{Medidas de Tendencia Central y Dispersión Muestrales}
        \begin{center}
            \begin{tabular}{|c|c|}
                \hline
                Medida Muestral & Carbs(g) \\
                \hline
                Media & 0.458142  \\
                $Q_1$ & 0.258305 \\
                $Q_2$ & 0.501985 \\
                $Q_3$ & 0.650737 \\
                Desviación Estándar & 0.233125 \\
                Mínimo & 0.012711 \\
                Máximo & 0.918259 \\
                Asimetría de Fisher & -0.226386 \\
                \hline
            \end{tabular}
        \end{center}

    \subsubsection{Comparativa con las Medidas Poblacionales}
        Si se compara con las medidas poblacionales, se puede observar que todos 
        los valores muestrales difieren de los mismos, en mayor o menor medida. 
        Esto refleja que el muestreo no es representativo para estimar las medidas 
        poblacionales. Siendo la asimetría el valor que más difiere debido a que es 
        un valor negativa; por lo que se observa distribuciones diferentes.
        \begin{center}
            \includegraphics[width=0.75\textwidth]{Resources/3_02_plot_01.png}
        \end{center}

\newpage

\end{document}