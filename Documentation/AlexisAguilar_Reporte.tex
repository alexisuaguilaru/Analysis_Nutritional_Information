\documentclass[12pt,a4paper]{article}

\usepackage{geometry}
\geometry{
    left=2cm, 
    right=2cm,
    top=3cm,  
    bottom=2cm
}

\usepackage[spanish,english]{babel}
\usepackage[utf8]{inputenc}
\usepackage{amsmath}

\usepackage{graphicx}
\usepackage{wrapfig}

\usepackage{csquotes}
\usepackage{hyperref}
\usepackage[style=ieee]{biblatex}
\addbibresource{referencias.bib}

\usepackage{setspace}
\setstretch{1.5}
\setlength{\parindent}{0pt}

\begin{document}
    \begin{titlepage}
        \begin{minipage}[c]{0.1\textwidth}
            \includegraphics[width=\textwidth]{Resources/logo_unam.jpg}
        \end{minipage}
        \begin{minipage}{0.8\textwidth}
            \centering
            {\Large\textbf{Universidad Nacional Autónoma de México}\\}
            {\large\textbf{Escuela Nacional de Estudios Superiores\\\underline{Unidad Morelia}}}
        \end{minipage}
        \begin{minipage}[c]{0.1\textwidth}
            \includegraphics[width=\textwidth]{Resources/logo_enes.jpg}
        \end{minipage}
        \vspace{3cm}

        \centering
        {\large{Reporte Final\\}}
        {\Large\textbf{Análisis de Valores Nutricionales por Tipo de Dieta}}
        \vspace{2cm}

        {{PRESENTA:\\}}
        {\large\textbf{Alexis Uriel Aguilar Uribe}}
        \vspace{1cm} 

        {{PROFESORES:\\}}
        {\large\textbf{Dra.\ María Del Río Francos}}\\
        {\large\textbf{Dr.\ César Andrés Torres Miranda}}
        \vspace{2cm}

        {{GRADO\\}}
        {\large\textbf{Licenciatura en Tecnologías para la Información en Ciencias}}
        \vspace{2cm}

        \flushleft{
        {\textbf{Asignatura:\ }Estadística Descriptiva e Inferencial}
        \vspace{2cm}}

        \flushright{
        {\textbf{A:\ }\underline{21 de Mayo del 2025}}}
        \vfill
    \end{titlepage}

\newpage

\tableofcontents

\newpage

\section{Introducción}
Este trabajo tiene como fin de exponer el proceso llevado a cabo para 
realizar el análisis estadístico de los valores nutricionales (macronutrientes) 
que aportan las dietas: \emph{DASH} (Dietary Approaches to Stop Hypertension), 
\emph{keto}, \emph{mediterránea}, \emph{paleo} (paleolítica) y \emph{vegana}.\\

Siendo el principal enfoque el responder si hay una diferencia nutricional 
significativa entre las diferentes dietas. En decir, hacer uso de 
técnicas de estadística descriptiva e inferencial para probar si existe 
una diferencia en los aportes nutricionales entre las distintas dietas que 
están siendo estudiadas. La anterior prueba se basa en recetas de diferentes 
cocinas a nivel mundial.

El propósito final del presenta trabajo es el de crear un modelo estadístico 
capaz de categorizar la dieta a la que pertenece una receta en base a los 
macronutrientes (carbohidratos, proteínas y grasas) que aporta.

\section{Objetivos Generales}
Para la realización de lo anterior expuesto, se puntualizan los objetivos del 
proyecto de manera incremental respecto al progreso general del mismo:
\begin{itemize}
    \item Realizar análisis estadístico de los macronutrientes en las diferentes. 
    Para una caracterización de los aportes nutricionales.
    \item Identificar la familia de distribuciones y sus respectivos parámetros 
    por cada macronutriente y dieta.
    \item Crear modelo estadístico para la categorización del tipo de dieta a 
    la que pertenece una receta en base a sus aportes nutricionales.
\end{itemize}

\newpage

\section{Marco Teórico}
La dieta es uno de los principales factores de riesgo de las enfermedades 
crónicas, y las enfermedades sensibles a la dieta contribuyen en gran medida 
a los costes sanitarios mundiales. Se han propuesto literalmente miles de 
``dietas'', que pueden describirse en términos generales como basadas en 
creencias, en alimentos específicos o en nutrientes; centradas en la 
pérdida de peso o en el aumento de peso (muscular); dietas de desintoxicación 
(detox) y dietas diseñadas por razones médicas específicas.\cite{marvastipopular} \\

Las ``dietas de moda'' son dietas populares durante un tiempo sin basarse 
necesariamente en una recomendación dietética estándar. A menudo promueven 
una pérdida de peso irracionalmente rápida o afirmaciones de salud sin 
sentido, y se anuncian como dietas que requieren poco esfuerzo por parte de 
quien las sigue. La promesa de ganancias fáciles, combinada con la presión 
social para lograr un determinado tipo de cuerpo, puede dejar al público 
susceptible a afirmaciones infundadas o exageradas.\cite{marvastipopular} \\

Las dietas estudiadas desde una perspectiva estadística en el presente 
trabajo, son englobadas en las ``dietas de moda'', que a veces son referidas 
como ``dietas sin evidencia científica''. Siendo  la dieta DASH la única 
que basada en fundamentos y evidencias.

\subsection{DASH (Dietary Approaches to Stop Hypertension)}
\cite{marvastipopular} La dieta DASH (Enfoques Dietéticos para Detener la 
Hipertensión) es un patrón dietético diseñado específicamente para ayudar 
a reducir la presión arterial y promover la salud general del corazón. Hace 
hincapié en el consumo de una variedad de alimentos ricos en nutrientes, 
como frutas, verduras, cereales integrales, proteínas magras y productos 
lácteos bajos en grasa, y en la limitación de la ingesta de sodio, grasas 
saturadas y azúcares añadidos. 

\subsection{Dieta Keto}
\cite{marvastipopular} Una dieta baja en hidratos de carbono (baja en 
carbohidratos) es un patrón alimentario que restringe la ingesta de 
carbohidratos, sustituyéndolos normalmente por mayores cantidades de 
proteínas y grasas. La dieta cetogénica es una forma de dieta baja en 
carbohidratos con un alto contenido en grasas en relación con la ingesta 
de proteínas y carbohidratos.\\

El objetivo de la dieta cetogénica es inducir la cetosis, un estado 
metabólico que se produce cuando el cuerpo quema grasa para obtener 
energía en lugar de glucosa, lo que induce la pérdida de peso.

\subsection{Dieta Mediterránea}
\cite{marvastipopular} La dieta mediterránea es un patrón alimentario 
inspirado en los hábitos alimenticios tradicionales de los países 
situados a orillas del mar Mediterráneo. Se caracteriza por un alto consumo 
de frutas, verduras, cereales integrales, legumbres, frutos secos y 
aceite de oliva; un consumo moderado de pescado y aves; y un bajo 
consumo de carnes rojas, alimentos procesados y dulces.\\

\subsection{Dieta Paleo (Paleolítica)}
\cite{marvastipopular} La dieta paleo, también conocida como dieta 
paleolítica o dieta del hombre de las cavernas, es un enfoque 
dietético que pretende imitar los hábitos alimentarios de nuestros 
antiguos antepasados del Paleolítico. \\

Hace hincapié en el consumo de alimentos integrales y no procesados 
que habrían estado al alcance de los primeros humanos, como carnes magras, 
pescado, frutas, verduras, frutos secos y semillas, y excluye los cereales, 
las legumbres, los productos lácteos, los alimentos procesados y los 
azúcares añadidos.

\subsection{Dieta Vegana}
\cite{marvastipopular} La dieta vegana es un patrón dietético basado en 
plantas que excluye el consumo de todos los productos de origen animal. Se 
centra en el consumo de una variedad de alimentos de origen vegetal, como 
frutas, verduras, cereales legumbres, frutos secos y semillas.\\

Es importante señalar que, aunque las dietas veganas pueden ser 
nutricionalmente adecuadas, debe prestarse atención a garantizar una 
ingesta suficiente de nutrientes esenciales como proteínas, hierro, 
calcio, vitamina B12 y ácidos grasos omega-3.

\newpage

\printbibliography[heading=bibintoc,title={Referencias Bibliográficas}]

\end{document}