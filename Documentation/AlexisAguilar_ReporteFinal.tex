\documentclass[12pt,a4paper]{article}

\usepackage{geometry}
\geometry{
    left=2cm, 
    right=2cm,
    top=3cm,  
    bottom=2cm
}

\usepackage[spanish]{babel}
\usepackage[utf8]{inputenc}
\usepackage{amsmath}

\usepackage{graphicx}
\usepackage{wrapfig}
\usepackage{makecell}
\usepackage{array,calc,xtab}

\usepackage{booktabs}
\usepackage{enumitem}
\usepackage[dvipsnames]{xcolor}

\usepackage{csquotes}
\usepackage[style=ieee,backend=biber]{biblatex}
\usepackage{hyperref}
\usepackage[noabbrev,capitalize]{cleveref}

\usepackage{setspace}

\addbibresource{referencias.bib}

\newcommand{\fullref}[1]{%
  \hyperref[#1]{\cref*{#1}~\nameref*{#1}}%
}
\newcommand{\anexoref}[1]{%
    \hyperref[#1]{Anexo~\ref*{#1} \nameref*{#1}}%
}
\crefname{section}{Sección}{Secciones}       
\crefname{subsection}{Subsección}{Subsecciones}
\crefname{figure}{Figura}{Figuras}           
\crefname{table}{Tabla}{Tablas}
\crefname{anexo}{Anexo}{Anexos}

\setstretch{1.25}
\setlength{\parindent}{0pt}

\begin{document}
    \begin{titlepage}
        \begin{minipage}[c]{0.1\textwidth}
            \includegraphics[width=\textwidth]{Resources/Cover/logo_unam.jpg}
        \end{minipage}
        \begin{minipage}{0.8\textwidth}
            \centering
            {\Large\textbf{Universidad Nacional Autónoma de México}\\}
            {\large\textbf{Escuela Nacional de Estudios Superiores\\\underline{Unidad Morelia}}}
        \end{minipage}
        \begin{minipage}[c]{0.1\textwidth}
            \includegraphics[width=\textwidth]{Resources/Cover/logo_enes.jpg}
        \end{minipage}
        \vspace{3cm}

        \centering
        {\large{Reporte Final\\}}
        {\Large\textbf{Análisis de Valores Nutricionales}}
        \vspace{2cm}

        {{PRESENTA:\\}}
        {\large\textbf{Alexis Uriel Aguilar Uribe}}
        \vspace{1cm} 

        {{PROFESORES:\\}}
        {\large\textbf{Dra.\ María Del Río Francos}}\\
        {\large\textbf{Dr.\ César Andrés Torres Miranda}}
        \vspace{2cm}

        {{GRADO\\}}
        {\large\textbf{Licenciatura en Tecnologías para la Información en Ciencias}}
        \vspace{2cm}

        \flushleft{
        {\textbf{Asignatura:\ }Estadística Descriptiva e Inferencial}
        \vspace{2cm}}

        \flushright{
        {\textbf{A:\ }\underline{26 de Mayo del 2025}}}
        \vfill
    \end{titlepage}

    \newpage

    \tableofcontents

    \newpage

    \section{Introducción}
    {
        Este trabajo tiene como fin de exponer el proceso llevado a cabo para 
        realizar el análisis estadístico de los valores nutricionales (macronutrientes) 
        que aportan las dietas: \emph{DASH} (Dietary Approaches to Stop Hypertension), 
        \emph{keto}, \emph{mediterránea}, \emph{paleo} (paleolítica) y \emph{vegana}.\\

        Siendo el principal enfoque el responder si hay una diferencia nutricional 
        significativa entre las diferentes dietas. En decir, hacer uso de 
        técnicas de estadística descriptiva e inferencial para probar si existe 
        una diferencia en los aportes nutricionales entre las distintas dietas que 
        están siendo estudiadas. La anterior prueba se basa en recetas de diferentes 
        cocinas a nivel mundial, y sobre éstas últimas serán auxiliares para realizar 
        un estudio más granulado sobre el comportamiento de las dietas en escenarios 
        más específicos.\\

        Los diferentes resultados y gráficas generadas que se presentan en las secciones 
        siguientes se encuentran disponibles en una libreta de \href{https://github.com/marimo-team/marimo}{Marimo} 
        en el siguiente recurso \href{https://github.com/alexisuaguilaru/Analysis_Nutritional_Information/blob/main/Documentation/AlexisAguilar_ReporteFinal.py}{Código para Reporte Final} 
        que pertenece al repositorio de código en GitHub para el presente proyecto.
    }

    \section{Objetivos Generales}
    {
        Para la realización de lo anterior expuesto, se puntualizan los objetivos del 
        proyecto:
        \begin{itemize}
            \item Realizar de un análisis estadístico de los macronutrientes en las 
            diferentes dietas con el fin de caracterizar sus aportes nutricionales y 
            sus distinciones en las diferentes cocinas.
            
            \item Conjeturar y probar hipótesis relacionadas a preguntas de interés 
            sobre los aportes nutricionales en cada dieta en base al análisis estadístico. 
            
            \item Probar si existe una diferencia significativa en los aportes 
            nutricionales entre las diferentes dietas con el fin de mostrar si cada 
            dieta del estudio es única.
        \end{itemize}
    }

    \newpage
 
    \newpage

    \section{Presentación de los Datos}
    {
        \subsection{Fuente de Datos}
        {
            El conjunto de datos se encuentra disponible en \cite{dataset_macronutrients}, 
            publicado por la comunidad de Kaggle. Los datos consisten de un conjunto 
            de recetas de diferentes dietas y cocinas, además incluye información de los macronutrientes que 
            aporta cada una.\\
            
            \cite{dataset_macronutrients} Del apartado de reconocimientos, se concluye que las recetas fueron 
            proporcionadas por diferentes creadores de las mismas y demás contribuidores 
            al conjunto de datos. 
            De los apartados de cómo usar el conjunto de datos e ideas de investigación, 
            se encuentra una idea, implícita, de la información que se quería estudiar. 
            La principal información de interés se vuelve que es: el crear planes 
            alimenticios saludables, y el estudiar la relación entre dieta y salud.
        }

        \subsection{Interés del Estudio}
        {
            Se consultó \cite{marvastipopular} en sus 
            capítulos 4 y 8, de donde se proporciona un entendimiento de la 
            importancia de los macronutrientes y una descripción general de las 
            dietas, resultando interesante que en cada dieta se 
            consumen diferentes alimentos y productos con ciertas características 
            para ya sea respetar alguna creencia, fundamento o cuota de macronutrientes. 
            De esto último, proporciona un indicio de que existe una diferencia entre 
            las dietas a nivel de aportes nutricionales, por lo tanto, lo que se 
            quiere realizar es probar esta diferencia haciendo 
            uso de la estadística.
        }

        \subsection{Variables del Conjunto de Datos}
        {
            El conjunto de datos consta de las siguientes variables y con $7806$ registros o instancias. Se menciona su 
            nombre, el tipo de variable y sus valores (en total y únicos):
            
            \begin{center}
                \begin{tabular}{r|llr}
                    \toprule
                    Variable & Nombre & Tipo & Valores Únicos\\
                    \midrule
                    1 & Diet\_type    & Cualitativa Nominal   &    5 \\
                    2 & Recipe\_name  & Cualitativa Nominal   & 7062 \\
                    3 & Cuisine\_type & Cualitativa Nominal   &   19 \\
                    4 & Protein       & Cuantitativa Continua & 6060 \\
                    5 & Carbs         & Cuantitativa Continua & 6618 \\
                    6 & Fat           & Cuantitativa Continua & 6322 \\
                    \bottomrule
                \end{tabular}
            \end{center}
            
            La variable \emph{Recipe\_Name} no es relevante para este trabajo pero figura 
            dentro del dataset. Se hace mención que el conjunto de datos no presenta 
            valores faltantes.
        }
    }

    \newpage

    \section{Estadística Descriptiva}\label{sec:eda}
    {
        \subsection{Preprocesamiento (Transformación) de los Datos}\label{subsec:trans_datos}
        {
            Los macronutrientes pueden tomar un amplio rango 
            de valores, esto puede generar un conflicto al momento de comparar 
            entre dietas. 
            Por ello, los valores de los macronutrientes de cada receta son 
            normalizados con la norma $l1$, y la norma calculada, el total de macronutrientes,
            se guarda como otra variable en \emph{Total\_macronutrients}.

            \begin{center}
                \begin{tabular}{l|rrrrr}
                \toprule
                    & dash & keto & mediterranean & paleo & vegan \\
                \midrule
                    kosher           & 5    & 0    & 0    & 2    & 0   \\
                    caribbean        & 3    & 7    & 1    & 6    & 1   \\
                    central europe   & 9    & 11   & 1    & 9    & 4   \\
                    japanese         & 9    & 10   & 2    & 5    & 24  \\
                    eastern europe   & 10   & 11   & 3    & 27   & 4   \\
                    middle eastern   & 21   & 17   & 26   & 12   & 15  \\
                    indian           & 20   & 12   & 3    & 9    & 48  \\
                    chinese          & 38   & 38   & 1    & 26   & 17  \\
                    asian            & 24   & 11   & 12   & 12   & 67  \\
                    south american   & 54   & 21   & 10   & 21   & 31  \\
                    south east asian & 31   & 34   & 8    & 29   & 46  \\
                    nordic           & 32   & 35   & 31   & 45   & 9   \\
                    mexican          & 61   & 60   & 17   & 48   & 38  \\
                    british          & 64   & 90   & 4    & 54   & 27  \\
                    world            & 234  & 6    & 6    & 3    & 10  \\
                    french           & 150  & 163  & 61   & 154  & 76  \\
                    italian          & 165  & 234  & 148  & 171  & 81  \\
                    mediterranean    & 176  & 89   & 1274 & 106  & 99  \\
                    american         & 639  & 663  & 145  & 535  & 925 \\
                \bottomrule
                \end{tabular}
            \end{center}

            Al considerar la cantidad de recetas que hay por dieta y cocina, se tiene que 
            hay configuraciones donde cuentan con pocas recetas, 
            por lo que para mitigar esta falta, lo que se realizar es juntar 
            los \emph{Cuisine\_type} en base a la cercanía geográfica, esto debido a 
            que si son colindantes comparten historia, cultura y, lo más relevante, ideas 
            gastronómicas. Por ello, las \emph{Cuisine\_type} se reagrupan de la siguiente 
            manera: 

            \begin{center}
                \begin{tabular}{l|l}
                \toprule
                    Grupos de \emph{Cuisine\_type} & \emph{Cuisine\_type} \\
                \midrule
                    american & american \\
                    mediterranean & mediterranean \\
                    world & world \\
                    latin american & mexican, south american, caribbean \\
                    european & \makecell{italian, french, nordic, eastern europe,\\central europe, kosher, british} \\
                    asian & \makecell{chinese, indian, south east asian,\\middle eastern, asian, japanese} \\
                \bottomrule
                \end{tabular}
            \end{center}
        }

        \subsection{Descripción de los Valores de las Variables}
        {
            \begin{itemize}[label=\textbullet]
            {
                \item \textbf{Diet\_type}: Variable nominal que representa el tipo de 
                dieta a la que pertenece una receta. Permite la estratificación principal de  
                las recetas y realizar hipótesis sobre lo qué está sucediendo en una dieta o entre las diferentes dietas.
                
                \item \textbf{Cuisine\_type}: Variable nominal que representa a qué 
                cocina o región pertenece una 
                receta. Permite el comparar cómo son las recetas 
                de una dieta en diferentes regiones.
                
                \item \textbf{Protein}: Variable continua que representa el 
                porcentaje, respecto al total de macronutrientes, de proteínas que son 
                aportados por una receta.

                \item \textbf{Carbs}: Variable continua que representa el 
                porcentaje, respecto al total de macronutrientes, de carbohidratos que 
                son aportados por una receta.
                
                \item \textbf{Fat}: Variable continua que representa el 
                porcentaje, respecto al total de macronutrientes, de grasas que son 
                aportados por una receta.
                
                \item \textbf{Total\_Macronutrients}: Variable continua que representa 
                el total de macronutrientes que son aportados por una receta. Esta variable 
                es auxiliar para prueba de hipótesis.
            }
            \end{itemize}
        }

        \subsection{Visión General de los Datos}
        {
            Primero se presenta un análisis sobre los macronutrientes de 
            las recetas sin estratificarlas según \emph{Diet\_type}:

            \begin{center}
                \begin{xtabular}{w{l}{\widthof{Desviación Estándar}}|rrr}
                \toprule
                    Medida & Carbs & Protein & Fat \\  
                \midrule
                    Media               & 0.433471 & 0.234762 & 0.331767 \\
                    $Q_1$               & 0.205251 & 0.110188 & 0.184583 \\
                    $Q_2$               & 0.432028 & 0.190931 & 0.314359 \\
                    $Q_3$               & 0.635058 & 0.338059 & 0.464532 \\
                    Desviación Estándar & 0.256032 & 0.163886 & 0.194920 \\
                    Mínimo              & 0.000330 & 0.000000 & 0.000000 \\
                    Máximo              & 1.000000 & 0.887557 & 0.997940 \\
                    Asimetría de Fisher & 0.189556 & 0.922401 & 0.461455 \\
                \bottomrule
                \end{xtabular}
            \end{center}

            Los valores en proteínas son bajos en comparación 
            con los carbohidratos y grasas si se hace uso de la mediana ($Q_2$), 
            dicho así. Esto es un indicio de que las recetas, en general, 
            tienden a ser altas en carbohidratos y, en menor medida, grasas; mientras que son bajas 
            en proteínas. Este último punto se apoya al considerar la asimetría de 
            las proteínas.\newline 

            Si se gráfica la distribución de los macronutrientes se tiene que, debido 
            a la asimetría y a la desviación estándar, contienen datos atípicos en proteínas 
            y grasas en una región positiva respecto a la mediana, y esto se relaciona con 
            lo mencionado de que una receta no tiende a un aporte alto de proteínas ni de grasas. 
            Por ello, se tiene que una receta es rica en carbohidratos y este patrón se 
            repite en todas las dietas.

            \begin{center}
                \includegraphics[width=0.85\textwidth]{Resources/EDA/VisionGeneral_1.png}

                \begin{xtabular}{w{l}{\widthof{latin american}}|w{r}{\widthof{dash}} w{r}{\widthof{keto}} w{r}{\widthof{mediterranean}} w{r}{\widthof{paleo}} w{r}{\widthof{vegan}}}
                \toprule
                    & dash & keto & mediterranean & paleo & vegan \\
                \midrule
                    american       & 639 & 663 & 145 & 535 & 925 \\
                    asian          & 143 & 122 & 52 & 93 & 217 \\
                    european       & 435 & 544 & 248 & 462 & 201 \\
                    latin american & 118 &  88 & 28 & 75 & 70 \\
                    mediterranean  & 176 &  89 & 1274 & 106 & 99 \\
                    world          & 234 &   6 & 6 & 3 & 10 \\
                \bottomrule
                \end{xtabular}
            \end{center}

            Al considerar la influencia de \emph{Cuisine\_type}, se tiene que las cajas y 
            bigotes (distribuciones) son diferentes (aunque comparten algunos rangos de valores) 
            y junto con la cantidad de datos, se podría esperar que tengan diferencias significativas. 
            En el tipo de cocina \emph{world} parece ser que tiene 
            un comportamiento atípico en comparación con las demás cocinas, por ello se va a 
            descartar del análisis.

            \begin{center}
                \includegraphics[width=0.9\textwidth]{Resources/EDA/VisionGeneral_2.png}
            \end{center}
        }

        \subsection{Estratificación de las Recetas por Dieta}
        {
            Estratificando las recetas según el tipo de dieta, se obtienen las siguientes 
            medidas por macronutriente:

            \begin{center}
               \begin{xtabular}{w{l}{\widthof{Desviación Estándar}}|w{r}{\widthof{-0.009756}} r w{r}{\widthof{mediterranean}} r w{r}{\widthof{-0.009756}}}
                \toprule
                    \multicolumn{6}{c}{Carbohidratos}\\
                \midrule
                    Medida & dash & keto & mediterranean & paleo & vegan \\
                \midrule
                    Media               & 0.491629 & 0.198449 & 0.422803 & 0.369912 & 0.593169 \\
                    $Q_1$               & 0.299070 & 0.085336 & 0.249507 & 0.191815 & 0.502866 \\
                    $Q_2$               & 0.502760 & 0.156879 & 0.438382 & 0.350726 & 0.625406 \\
                    $Q_3$               & 0.683241 & 0.264915 & 0.605733 & 0.512822 & 0.713660 \\
                    Desviación Estándar & 0.249919 & 0.155979 & 0.212640 & 0.219894 & 0.171149 \\
                    Mínimo              & 0.001526 & 0.002060 & 0.006733 & 0.003612 & 0.000330 \\
                    Máximo              & 1.000000 & 1.000000 & 0.992746 & 0.976322 & 0.986872 \\
                    Asimetría de Fisher & -0.009756 & 1.562095 & -0.123218 & 0.473617 & -0.735666 \\
                \bottomrule
                \end{xtabular} 
            \end{center}
            
            \begin{center}
               \begin{xtabular}{w{l}{\widthof{Desviación Estándar}}|rrrrr}
                \toprule
                    \multicolumn{6}{c}{Proteínas}\\
                \midrule
                    Medida & dash & keto & mediterranean & paleo & vegan \\
                \midrule
                    Media               & 0.220058 & 0.302718 & 0.280190 & 0.250207 & 0.148801 \\
                    $Q_1$               & 0.094259 & 0.159164 & 0.159930 & 0.103933 & 0.085853 \\
                    $Q_2$               & 0.180192 & 0.304214 & 0.229058 & 0.207286 & 0.139807 \\
                    $Q_3$               & 0.311076 & 0.410555 & 0.377918 & 0.375863 & 0.190724 \\
                    Desviación Estándar & 0.161171 & 0.166667 & 0.162485 & 0.174915 & 0.086151 \\
                    Mínimo              & 0.000000 & 0.000000 & 0.005036 & 0.000000 & 0.001921 \\
                    Máximo              & 0.833467 & 0.856868 & 0.887557 & 0.858503 & 0.647416 \\
                    Asimetría de Fisher & 1.029402 & 0.313745 & 0.962969 & 0.710366 & 1.439623 \\
                \bottomrule
                \end{xtabular} 
            \end{center}

            \begin{center}
               \begin{xtabular}{w{l}{\widthof{Desviación Estándar}}|rrrrr}
                \toprule
                    \multicolumn{6}{c}{Grasas}\\
                \midrule
                    Medida & dash & keto & mediterranean & paleo & vegan \\
                \midrule
                    Media               & 0.288313 & 0.498833 & 0.297007 & 0.379881 & 0.258029 \\
                    $Q_1$               & 0.159870 & 0.408251 & 0.180738 & 0.257510 & 0.143034 \\
                    $Q_2$               & 0.267363 & 0.506381 & 0.268950 & 0.382872 & 0.231985 \\
                    $Q_3$               & 0.393803 & 0.592312 & 0.390790 & 0.488153 & 0.344991 \\
                    Desviación Estándar & 0.182336 & 0.165075 & 0.160349 & 0.174737 & 0.160262 \\
                    Mínimo              & 0.000000 & 0.000000 & 0.001731 & 0.006210 & 0.000112 \\
                    Máximo              & 0.973404 & 0.997940 & 0.968722 & 0.968835 & 0.994887 \\
                    Asimetría de Fisher & 0.766955 & -0.120900 & 0.878710 & 0.325286 & 1.092007 \\
                \bottomrule
                \end{xtabular} 
            \end{center}

            Al considerar las medias de los diferentes macronutrientes y dietas, se tiene 
            que siguen diferentes patrones, es decir, cada dieta propicia que ciertos macronutrientes 
            sean más dominantes en las recetas, como también qué tanto dominan esos valores al 
            considerar la asimetría. Por ello, se podría esperar que las dietas 
            tengan o sigan distribuciones en sus macronutrientes distintas.
            Un análisis más granular se presenta en \anexoref{anexo:C}
        }
    }

    \newpage

    \section{Análisis Bivariado}\label{sec:biva}
    {
        El realizar un análisis bivariado usando todas las dietas va a 
        provocar una desvanecimiento de la información debido a que las dietas 
        tienen comportamientos heterogéneas; 
        donde esto se muestra en \fullref{sec:eda}. Debido a que los valores 
        de los macronutrientes suman $1$ se tiene que el incrementar 
        un macronutriente hace que los otros decrezcan, por lo que el 
        valor de la correlación refleja esta fuerza en que decrece o, 
        en ciertos casos, crece.\newline 
        
        En la mayoría de los casos se tienen 
        correlaciones mayores a $0.5$ en valor absoluto, esto representa 
        que son tendencias fuertes que tienen las diferentes dietas sobre 
        la composiciones de sus macronutrientes. Se tiene que las relaciones 
        entre carbohidratos contra proteínas y grasas permiten describir de 
        otra manera cada dieta. \anexoref{anexo:B.1}

        \begin{center}
            \begin{tabular}{lllrr}
            \toprule
                Dieta & Macronutrientes & Centroide & Covarianza & \makecell{Coeficiente\\Correlación} \\
            \midrule
                dash          & (\textcolor{ForestGreen}{Carbs} , \textcolor{BrickRed}{Protein}) & [0.4916 0.2200] & -0.0275 & -0.6850 \\
                dash          & (\textcolor{ForestGreen}{Carbs} , \textcolor{YellowOrange}{Fat})     & [0.4916 0.2883] & -0.0348 & -0.7650 \\
                dash          & (\textcolor{BrickRed}{Protein}  , \textcolor{YellowOrange}{Fat})   & [0.2200 0.2883] &  0.0016 &  0.0550 \\
                keto          & (\textcolor{ForestGreen}{Carbs} , \textcolor{BrickRed}{Protein}) & [0.1984 0.3027] & -0.0124 & -0.4780 \\
                keto          & (\textcolor{ForestGreen}{Carbs} , \textcolor{YellowOrange}{Fat})     & [0.1984 0.4988] & -0.0119 & -0.4621 \\
                keto          & (\textcolor{BrickRed}{Protein}  , \textcolor{YellowOrange}{Fat})   & [0.3027 0.4988] & -0.0153 & -0.5578 \\
                mediterranean & (\textcolor{ForestGreen}{Carbs} , \textcolor{BrickRed}{Protein}) & [0.4228 0.2801] & -0.0229 & -0.6643 \\
                mediterranean & (\textcolor{ForestGreen}{Carbs} , \textcolor{YellowOrange}{Fat})     & [0.4228 0.2970] & -0.0222 & -0.6529 \\
                mediterranean & (\textcolor{BrickRed}{Protein}  , \textcolor{YellowOrange}{Fat})   & [0.2801 0.2970] & -0.0034 & -0.1323 \\
                paleo         & (\textcolor{ForestGreen}{Carbs} , \textcolor{BrickRed}{Protein}) & [0.3699 0.2502] & -0.0242 & -0.6293 \\
                paleo         & (\textcolor{ForestGreen}{Carbs} , \textcolor{YellowOrange}{Fat})     & [0.3699 0.3798] & -0.0241 & -0.6284 \\
                paleo         & (\textcolor{BrickRed}{Protein}  , \textcolor{YellowOrange}{Fat})   & [0.2502 0.3798] & -0.0063 & -0.2089 \\
                vegan         & (\textcolor{ForestGreen}{Carbs} , \textcolor{BrickRed}{Protein}) & [0.5931 0.1488] & -0.0055 & -0.3740 \\
                vegan         & (\textcolor{ForestGreen}{Carbs} , \textcolor{YellowOrange}{Fat})     & [0.5931 0.2580] & -0.0237 & -0.8668 \\
                vegan         & (\textcolor{BrickRed}{Protein}  , \textcolor{YellowOrange}{Fat})   & [0.1488 0.2580] & -0.0019 & -0.1381 \\
            \bottomrule
            \end{tabular}
        \end{center}
        
        La tabla anterior permite observar con mayor detenimiento que todos los valores 
        son diferentes. Con ello, se tiene que 
        los macronutrientes tendrán diferentes comportamientos según la dieta a la que 
        pertenezcan. Esto implica que las dietas van a favorecer más a ciertas combinaciones 
        entre macronutrientes, y esto se relaciona con los alimentos y productos que son 
        consumidos dentro de cada dieta. Por lo tanto, se 
        muestra que se puede determinar cuál es la dieta que favorece más un cierto macronutriente 
        más que las otras; esto se realiza al considerar justamente los coeficientes de correlación.
    }

    \newpage

    \section{Muestreo e Intervalos de Confianza}
    {
        Como las tres variables cuantitativas tienen el mismo nivel 
        de relevancia, se opta por usar los \emph{Carbs} como atributo 
        para el muestreo. Y para ambos muestreos se realizan de 
        tamaño $50$ y, usando la Regla de Sturges, se emplean 7 clases 
        o bins para la tabla de frecuencias.

        \subsection{Muestreo Simple Aleatorio}\label{subsec:sample_rand}
        {
            Al considerar los estadísticos muestrales se puede 
            apreciar que difieren en todos las medidas, por lo que este muestreo 
            no es representativo del conjunto de datos, es decir, siguen diferentes 
            distribuciones principalmente al usar la media y los cuartiles. Por lo 
            tanto, o se puede cambiar de estrategia o usar un tamaño de muestreo 
            más grande.

            \begin{center}
                \begin{tabular}{ccccc}
                        \multicolumn{5}{c}{Resultados del Muestreo} \\ 
                    \midrule
                        0.780289 &   0.15143502 & 0.13749852 &  0.60841622 & 0.69139498 \\ 
                        0.04172702 & 0.44570519 & 0.74317898 & 0.53322363 & 0.23569666\\
                        0.20240863 & 0.55306465 & 0.98825124 &  0.62201877 & 0.08367046 \\ 
                        0.7226847 & 0.53068409 & 0.10743335 & 0.06312504 & 0.75586 \\
                        0.69076424 & 0.18063042 & 0.14876349 &  0.76112551 & 0.61451786 \\ 
                        0.81901361 & 0.61532896 & 0.22294326 & 0.58007183 & 0.66899583\\
                        0.69092336 & 0.13819862 & 0.08259526 &  0.37709587 & 0.17925896 \\ 
                        0.48040837 & 0.98382353 & 0.49527076 & 0.1451375 &  0.19026882\\
                        0.63525297 & 0.44261674 & 0.22074399 &  0.41293142 & 0.60653006 \\ 
                        0.55742063 & 0.42509449 & 0.19835071 & 0.45468297 & 0.3959553\\
                \end{tabular}
            \end{center}

            \begin{center}
                \begin{tabular}{lrrrr}
                \toprule
                    \makecell{Marca de\\Clase} & \makecell{Frecuencia\\Abosluta} & \makecell{Frecuencias\\Relativa} & \makecell{Frecuencia\\Acumuladas} & z-score \\
                \midrule
                    0.109336 & 10 & 0.20 & 0.20 & -1.308939 \\
                    0.244554 & 8  & 0.16 & 0.36 & -0.786707 \\
                    0.379771 & 6  & 0.12 & 0.48 & -0.264474 \\
                    0.514989 & 8  & 0.16 & 0.64 &  0.257759 \\
                    0.650207 & 10 & 0.20 & 0.84 &  0.779991 \\
                    0.785425 & 6  & 0.12 & 0.96 &  1.302224 \\
                    0.920642 & 2  & 0.04 & 1.00 &  1.824456 \\
                \bottomrule
                \end{tabular}
            \end{center}

            \begin{minipage}{0.35\textwidth}
                \centering
                \begin{tabular}{l|r}
                \toprule
                    Medida Muestral & Carbs \\
                \midrule
                    Media               & 0.448250 \\
                    $Q_1$               & 0.192289 \\
                    $Q_2$               & 0.467546 \\
                    $Q_3$               & 0.631944 \\
                    Desviación Estándar & 0.258922 \\
                    Mínimo              & 0.041727 \\
                    Máximo              & 0.988251 \\
                    Asimetría de Fisher & 0.088215 \\
                \bottomrule
                \end{tabular}
            \end{minipage}%
            \begin{minipage}{0.65\textwidth}
                \centering
                \includegraphics[width=0.9\textwidth]{Resources/Sampling/Random.png}
            \end{minipage}
        }

        \subsection{Muestreo Aleatorio Estratificado}
        {
            Al igual que en \fullref{subsec:sample_rand}, se tiene que en el muestreo 
            estratificado no es representativo del conjunto de datos, esto se deriva 
            del hecho de que las medidas muestrales difieren notoriamente de las poblacionales 
            y además de seguir una distribución diferente. Con ello, se tiene que la solución 
            para obtener un muestreo representativo se vuelve justamente incrementar 
            el tamaño de la muestra.

            \begin{center}
                \begin{tabular}{ccccc}
                        \multicolumn{5}{c}{Resultados del Muestreo} \\ 
                    \midrule
                        \textcolor{Mulberry}{0.98825124} & \textcolor{Mulberry}{0.44840909} & \textcolor{Mulberry}{0.89960239} & \textcolor{Mulberry}{0.83874942} & \textcolor{Mulberry}{0.365615}   \\
                        \textcolor{Mulberry}{0.49887176} & \textcolor{Mulberry}{0.58630187} & \textcolor{Mulberry}{0.3157274}  & \textcolor{Mulberry}{0.5842796}  & \textcolor{Mulberry}{0.57451464} \\
                        \textcolor{BurntOrange}{0.08873597} & \textcolor{BurntOrange}{0.68983711} & \textcolor{BurntOrange}{0.16884721} & \textcolor{BurntOrange}{0.15916624} & \textcolor{BurntOrange}{0.09565607} \\
                        \textcolor{BurntOrange}{0.52595787} & \textcolor{BurntOrange}{0.07693278} & \textcolor{BurntOrange}{0.27939369} & \textcolor{BurntOrange}{0.12009859} & \textcolor{BurntOrange}{0.22105597} \\
                        \textcolor{Cyan}{0.45648684} & \textcolor{Cyan}{0.56049479} & \textcolor{Cyan}{0.57414311} & \textcolor{Cyan}{0.28213503} & \textcolor{Cyan}{0.61598047} \\
                        \textcolor{Cyan}{0.56339749} & \textcolor{Cyan}{0.52937621} & \textcolor{Cyan}{0.5663925}  & \textcolor{Cyan}{0.65277566} & \textcolor{Cyan}{0.2666113}  \\
                        \textcolor{Cyan}{0.13609793}     & \textcolor{Cyan}{0.41432132}     & \textcolor{Mahogany}{0.13118082} & \textcolor{Mahogany}{0.07637211} & \textcolor{Mahogany}{0.17604633} \\
                        \textcolor{Mahogany}{0.77068989} & \textcolor{Mahogany}{0.52892388} & \textcolor{Mahogany}{0.39944097} & \textcolor{Mahogany}{0.13687658} & \textcolor{Mahogany}{0.00708689} \\
                        \textcolor{PineGreen}{0.6840657}  & \textcolor{PineGreen}{0.32537609} & \textcolor{PineGreen}{0.3360161}  & \textcolor{PineGreen}{0.50114789} & \textcolor{PineGreen}{0.89117851} \\
                        \textcolor{PineGreen}{0.65409493} & \textcolor{PineGreen}{0.30271941} & \textcolor{PineGreen}{0.80744214} & \textcolor{PineGreen}{0.60015393} & \textcolor{PineGreen}{0.65224341} \\
                \end{tabular}
            \end{center}

            Leyenda de colores: \textcolor{Mulberry}{DASH}, \textcolor{BurntOrange}{Keto}, 
            \textcolor{Cyan}{Mediterránea}, \textcolor{Mahogany}{Paleo} y \textcolor{PineGreen}{Vegana}.

            \begin{center}
                \begin{tabular}{lrrrr}
                \toprule
                    \makecell{Marca de\\Clase} & \makecell{Frecuencia\\Abosluta} & \makecell{Frecuencias\\Relativa} & \makecell{Frecuencia\\Acumuladas} & z-score \\
                \midrule
                    0.077170 &  9 & 0.18 & 0.18 & -1.457119 \\
                    0.217336 &  7 & 0.14 & 0.32 & -0.898073 \\
                    0.357503 &  7 & 0.14 & 0.46 & -0.339028 \\
                    0.497669 & 10 & 0.20 & 0.66 & 0.220017 \\
                    0.637835 & 11 & 0.22 & 0.88 & 0.779062 \\
                    0.778002 &  3 & 0.06 & 0.94 & 1.338107 \\
                    0.918168 &  3 & 0.06 & 1.00 & 1.897152 \\
                \bottomrule
                \end{tabular}
            \end{center}

            \begin{minipage}{0.35\textwidth}
                \centering
                \begin{tabular}{l|r}
                \toprule
                    Medida Muestral & Carbs \\
                \midrule
                    Media               & 0.442505 \\
                    $Q_1$               & 0.232445 \\
                    $Q_2$               & 0.477679 \\
                    $Q_3$               & 0.596691 \\
                    Desviación Estándar & 0.250725 \\
                    Mínimo              & 0.007087 \\
                    Máximo              & 0.988251 \\
                    Asimetría de Fisher & 0.129906 \\
                \bottomrule
                \end{tabular}
            \end{minipage}%
            \begin{minipage}{0.65\textwidth}
                \centering
                \includegraphics[width=0.9\textwidth]{Resources/Sampling/Stratified.png}
            \end{minipage}
        }
    
        \subsection{Intervalos de Confianza}
        {
            En cada muestreo, se determina los intervalos 
            con los niveles de confianza del $85\%$, $95\%$ y $99\%$, 
            y en cada intervalo se determinó si la media poblacional de 
            los carbohidratos pertenece al intervalo construido. De los 
            intervalos de confianza construidos, se tiene que en todos
            pertenece la media poblacional (como era de esperarse).\\

            \begin{minipage}{0.5\textwidth}
                \centering
                \begin{tabular}{r|rr}
                    \multicolumn{3}{c}{Muestreo Simple Aleatorio} \\
                \midrule
                    \makecell{Nivel de\\Confianza} & \makecell{Límite\\Inferior} & \makecell{Límite\\Superior} \\
                \midrule
                    $85\%$ & 0.395538 & 0.500961 \\
                    $95\%$ & 0.376481 & 0.520018 \\
                    $99\%$ & 0.353930 & 0.542569 \\
                \bottomrule
                \end{tabular}
            \end{minipage}% 
            \begin{minipage}{0.5\textwidth}
                \centering
                \begin{tabular}{r|rr}
                    \multicolumn{3}{c}{Muestreo Aleatorio Estratificado} \\
                \midrule
                    \makecell{Nivel de\\Confianza} & \makecell{Límite\\Inferior} & \makecell{Límite\\Superior} \\
                \midrule
                    $85\%$ & 0.391463 & 0.493548 \\
                    $95\%$ & 0.373009 & 0.512001 \\
                    $99\%$ & 0.351172 & 0.533839 \\
                \bottomrule
                \end{tabular}
            \end{minipage}
        }
    }

    \newpage

    \section{Pruebas de Hipótesis}
    {
        Para cada una de las pruebas se hace uso de una significancia de $\alpha = 0.05$, 
        para las diferentes pruebas se reportan los p valores (p-value) y el valor del 
        estadístico que se obtuvieron, en el caso de que sean necesarios.

        \subsection{Normalidad y Transformaciones Box-Cox}
        {
            Debido a que las distribuciones de los macronutrientes solo pueden tomar valores 
            en el rango $[0,1]$, se vuelve que al aplicar alguna transformación se genera una 
            pérdida de información y de interpretación de los datos, además	que se pueden 
            perder relaciones entre los macronutrientes. Por ello, no se puede 
            hacer uso de estas transformaciones, pero si se puede caracterizar las distribuciones 
            de los macronutrientes haciendo uso la familia de distribuciones beta \cite{beta_distribution}, 
            que son distribuciones no normales.
        }
    
        \subsection{Dieta DASH}
        {
            Se quiere probar si esta dieta se encuentra 
            en balance nutricional, es decir, si se llega a consumir la 
            misma proporción de macronutrientes diariamente. 
            Se realiza el supuesto de que en un día normal se consumen 
            cinco comidas, el equivalente a cinco recetas, por lo que 
            se muestrean 50 días siguiendo esta dieta (o 250 recetas), 
            para después determinar las proporciones 
            de macronutrientes que son consumidos en cada día.\newline

            Con las muestras de las proporciones de macronutrientes se realiza 
            un conjunto de hipótesis por cada macronutriente, teniendo lo 
            siguiente:
            \begin{align*}
                H_0 :& \overline{x}_M = 1/3  \\
                H_1 :& \overline{x}_M \ne 1/3 
            \end{align*}
        
            Donde $M$ se refiere a uno de los tres macronutrientes con los que 
            se cuenta. Se hace uso de la Prueba t para determinar si existe 
            una diferencia significativa respecto al balance. Se obtienen 
            los siguientes resultados:

            \begin{center}
                \begin{tabular}{l|rr}
                \toprule
                    Macronutriente & Valor P & Estadístico T \\
                \midrule
                    Carbs   & 0.0000 &  7.3856 \\
                    Protein & 0.0000 & -9.2252 \\
                    Fat     & 0.0816 & -1.7778 \\
                \bottomrule
                \end{tabular}
            \end{center}

            Se tiene que las grasas es el único macronutriente que se encuentra en 
            balance pero aún así tiene un tendencia a ser menor al balance nutricional; 
            mientras que los otros macronutrientes no se encuentran en este balance.
             Esto hace 
            de la dieta DASH una que no tiene un balance nutricional, e invalidando que 
            sea saludable.
        }

        \subsection{Dieta Keto}
        {
            Se quiere probar si en todas las cocinas de esta dieta 
            se verifica que el consumo de carbohidratos es menor al de grasas. 
            El conjunto de hipótesis son de la forma:
            \begin{align*}
                H_0 : \overline{c}_C = \overline{f}_C \\
                H_1 : \overline{c}_C < \overline{f}_C
            \end{align*}

            Donde $\overline{c}_C$ y $\overline{f}_C$ representan 
            la media de los carbohidratos y grasas, respectivamente, de 
            una cocina $C$. Se realiza la Prueba t junto con la Prueba de Levenne para 
            probar la homocedasticidad entre los macronutrientes. Se 
            obtienen los siguientes resultados:

            \begin{center}
                \begin{tabular}{l|rr}
                \toprule
                    Cocina & Valor P & Estadístico T \\
                \midrule
                    american       & 0.0000 & -35.6024 \\
                    asian          & 0.0000 &  -8.1913 \\
                    european       & 0.0000 & -34.4175 \\
                    latin american & 0.0000 & -10.6138 \\
                    mediterranean  & 0.0000 & -11.6289 \\
                \bottomrule
                \end{tabular}
            \end{center}

            Al observar los valores p, se tiene que en todas las cocinas se cumple 
            que la ingesta de grasas es mayor que la de carbohidratos, por lo 
            tanto, las diferentes cocinas son representativas de la dieta keto.
        }

        \subsection{Dieta Mediterránea}

        {
            Se quiere probar si existe una diferencia en el comportamiento, aportes 
            nutricionales, entre las recetas de la propia región del Mediterráneo en contraste 
            con las recetas de las demás cocinas. Se tiene el siguiente conjunto de hipótesis:
            \begin{align*}
                H_0 : F^M_m = F^O_m \\
                H_1 : F^M_m \ne F^O_m  
            \end{align*}

            Donde $F^M_m$ y $F^O_m$ representan la distribución del macronutriente $m$ sobre las 
            recetas pertenecientes al mediterráneo, $M$, y de las que no pertenecen, $O$, respectivamente. 
            Se hace uso de la Prueba Kolmogorov-Smirnov para detectar cualquier diferencia entre las distribuciones.
            Se obtienen los siguientes resultados:

            \begin{center}
                \begin{tabular}{l|rr}
                \toprule
                    Macronutriente & Valor P & Estadístico D \\
                \midrule
                    Carbs   & 0.3476 & 0.0497 \\
                    Protein & 0.0018 & 0.0998 \\
                    Fat     & 0.7416 & 0.0361 \\
                \bottomrule
                \end{tabular}
            \end{center}
            
            De los valores p, se tiene que los carbohidratos y las grasas 
            no tienen una diferencia significa, por lo que siguen la misma 
            distribución y, por lo tanto, las recetas tendrán composiciones 
            similares en estos macronutrientes. En cambio, en las proteínas 
            ocurre lo contrario, mostrando que las regiones influyen sobre 
            los alimentos de origen animal que emplean.
        }

        \subsection{Dieta Paleo}
        {
            Se quire probar como las regiones geográficas influyen o contribuyen 
            a los aportes de proteínas en las recetas pertenecientes a una cocina. 
            Se tiene el siguiente conjunto de hipótesis:
            \begin{align*}
                H_0 :& \text{Todas las medianas son iguales} \\
                H_1 :& \text{Por al menos una de las medianas es diferente}
            \end{align*}

            Donde las medianas se refieren a las de proteínas por cocina. Para probar 
            estas hipótesis se hace uso de la Prueba de Kruskal-Wallis junto con la 
            Prueba de Dunn para probar en qué cocinas, específicamente, difieren. 
            Se obtienen los 
            siguientes resultados: el valor p es $0.0000349902$ y el estadístico H es 
            $25.783473$. Aplicando la prueba de Dunn, se tienen los siguientes valores p:

            \begin{center}
                \begin{tabular}{l|rrrrr}
                \toprule
                 & american & asian & european & latin american & mediterranean \\
                \midrule
                american       & 1 & 0.000007 & 1 & 1 & 1 \\
                asian          & 0.000007 & 1 & 0.000096 & 0.006961 & 0.000319 \\
                european       & 1 & 0.000096 & 1 & 1 & 1 \\
                latin american & 1 & 0.006961 & 1 & 1 & 1 \\
                mediterranean  & 1 & 0.000319 & 1 & 1 & 1 \\
                \bottomrule
                \end{tabular}
            \end{center}

            Se tiene que la dieta asiática es la única que difiere significativamente 
            de las demás en las proteínas, haciendo que sea una cocina con recetas 
            más ricas en proteínas que en las demás, esto se podría explicar por 
            medio de un mejor aprovechamiento de los recursos pesqueros e hídricos.
        }

        \subsection{Dieta Vegana}
        {
            Se quiere probar si los aportes de carbohidratos son mayores 
            que a los de proteínas en cada una de las cocinas. Se tiene el 
            siguiente conjunto de hipótesis:
            \begin{align*}
                H_0 : \overline{p}_C = \overline{c}_C \\
                H_1 : \overline{p}_C < \overline{c}_C
            \end{align*}

            De donde $\overline{p}_C$ y $\overline{c}_C$ representan la media 
            de las proteínas y carbohidratos, respectivamente, de una cocina $C$. 
            Se realiza la Prueba t junto con la Prueba de Levenne para 
            probar la homocedasticidad entre los macronutrientes. Se obtienen 
            los siguientes resultados:

            \begin{center}
                \begin{tabular}{l|rr}
                \toprule
                    Cocina & Valor P & Estadístico T \\
                \midrule
                    american       & 0 & -70.993397 \\
                    asian          & 0 & -28.727771 \\
                    european       & 0 & -35.549140 \\
                    latin american & 0 & -27.142548 \\
                    mediterranean  & 0 & -24.670579 \\
                \bottomrule
                \end{tabular}
            \end{center}

            De los valores p, se tiene que en todas las dietas la ingesta de 
            proteínas es menor que la de carbohidratos, esto representa que 
            es una dieta baja de proteínas y que se tiene que complementar 
            con una fuente externa rica en proteínas.
        }

        \subsection{Diferencias entre Dietas}
        {
            Se quiere probar si
            existe una diferencia en los aportes nutricionales entre las diferentes 
            dietas. Para ello, se hace uso de las distribuciones de los macronutrientes 
            que presentan las diferentes dietas. Se tiene el siguiente conjunto de  
            hipótesis:
            \begin{align*}
                H_0 : F_m(x) = G_m(x) \\
                H_1 : F_m(x) \ne G_m(x)
            \end{align*}

            Donde $F_m(x)$ y $G_m(x)$ son las distribuciones del macronutriente $m$ en 
            dos dietas diferentes. Se hace uso de la Prueba de Kolmogorov-Smirnov. Para 
            cuantificar la diferencia entre dietas 
            se promedian los resultados del estadístico calculado en cada una de las pruebas.
            Se obtienen los siguientes resultados.

            \begin{center}
                \begin{tabular}{l|rrrrr}
                \toprule
                    \multicolumn{6}{c}{Diferencias Significativas} \\
                \midrule
                    & dash & keto & mediterranean & paleo & vegan \\
                \midrule
                    dash & 0 & 3 & 3 & 3 & 3 \\
                    keto & 3 & 0 & 3 & 3 & 3 \\
                    mediterranean & 3 & 3 & 0 & 3 & 3 \\
                    paleo & 3 & 3 & 3 & 0 & 3 \\
                    vegan & 3 & 3 & 3 & 3 & 0 \\
                \bottomrule
                \end{tabular}
            \end{center}
            
            \begin{center}
                \begin{tabular}{l|rrrrr}
                \toprule
                    \multicolumn{6}{c}{Promedio de los Estadísticos D} \\
                \midrule
                    & dash & keto & mediterranean & paleo & vegan \\
                \midrule
                    dash          & 0.000000 & 0.450111 & 0.147704 & 0.207993 & 0.206442 \\
                    keto          & 0.450111 & 0.000000 & 0.391530 & 0.299025 & 0.624274 \\
                    mediterranean & 0.147704 & 0.391530 & 0.000000 & 0.198158 & 0.282763 \\
                    paleo         & 0.207993 & 0.299025 & 0.198158 & 0.000000 & 0.394270 \\
                    vegan         & 0.206442 & 0.624274 & 0.282763 & 0.394270 & 0.000000 \\
                \bottomrule
                \end{tabular}
            \end{center}

            Se tiene que todas las dietas son diferentes entre sí, esto lleva a que 
            los propios estilos de cocina, los ingredientes que usan y las preparaciones 
            sean lo suficientemente diversas y únicas para que no se parezcan entre sí. 
            Y al usar el promedio del estadístico, se tiene que no hay dietas que se acerquen 
            mucho a otras, es decir, que se parezcan, siendo la dieta DASH y mediterránea las 
            que más se parecen, mientras que la dieta vegana y la keto son opuestas de la otra. 
        }

        \subsection{Interacción entre Dietas y Cocina}
        {
            Se quiere probar si las dietas y las cocinas (en conjunto) tienen una influencia 
            sobre los diferentes macronutrientes. Para ello se realiza una ANOVA de dos vías 
            no paramétricas, por medio de la Prueba de Scheirer-Ray-Hare, los resultados que 
            se obtienen por macronutriente y fuente de variación son:

            \begin{center}
                \begin{tabular}{l|rr}
                \toprule
                    \multicolumn{3}{c}{Carbs} \\
                \midrule
                    Fuente & Valor P & Estadístico F \\
                \midrule
                    Dieta               & 0.0000 & 796.5672 \\
                    Cocina              & 0.0039 &   3.8450 \\
                    Dieta $\ast$ Cocina & 0.0153 &   1.9108 \\
                \bottomrule
                \end{tabular}
                \vspace*{0.5cm}
                \begin{tabular}{l|rr}
                    \toprule
                    \multicolumn{3}{c}{Protein} \\
                \midrule
                    Fuente & Valor P & Estadístico F \\
                \midrule
                    Dieta               & 0.0000 & 220.3292 \\
                    Cocina              & 0.0000 &  30.4022 \\
                    Dieta $\ast$ Cocina & 0.0052 &   2.1329 \\
                \bottomrule
                \end{tabular}\\
                \vspace*{0.1cm}
                \begin{tabular}{l|rr}
                    \toprule
                    \multicolumn{3}{c}{Fat} \\
                \midrule
                    Fuente & Valor P & Estadístico F \\
                \midrule
                    Dieta               & 0.0000   & 523.5090 \\
                    Cocina              & 0.0000   &   8.7621 \\
                    Dieta $\ast$ Cocina & 0.000070 &   2.9393 \\
                \bottomrule
                \end{tabular}
            \end{center}

            De manera individual, las dietas y las cocinas si tienen una influencia 
            sobre los valores en los macronutrientes de manera contundente, esto 
            coincide con las conclusiones del apartado anterior además de que las dietas
            deberían que tender a tener diferencias.\newline
            
            Cuando se considera su interacción, 
            se tiene que en las proteínas y grasas tienen una interacción fuerte debido 
            a sus valores, además de ser los macronutrientes que más fluctúan cuando 
            se comparan entre dietas y cocinas. En carbohidratos, se tiene que esta 
            influencia se reduce, esto se relaciona a que los carbohidratos son más 
            estables entre las dietas y cocinas, haciendo que tomen rango de valores 
            parecidos, dicho así, son más dominantes sobre los otros macronutrientes.
        }

        \subsection{Regresión Lineal}
        {
            No se realiza el ajuste de curvas sin estratificar debido a lo mostrado 
            en \fullref{sec:eda}, donde, principalmente, la información es desvanecida 
            al considerar todas las dietas.\newline
        
            Se quiere probar si la dependencia lineal de los macronutrientes en cada 
            dieta es significativa, para ello se hace uso de la prueba de Correlación 
            de Pearson. Se tienen los siguientes conjunto de hipótesis:\\
            
            \begin{minipage}{0.5\textwidth}
                \begin{align*}
                    H_0 :& \rho = 0 \\
                    H_1 :& \rho < 0
                \end{align*}
            \end{minipage}%
            \hfill
            \begin{minipage}{0.5\textwidth}
                \begin{align*}
                    H_0 :& \rho = 0 \\
                    H_1 :& \rho > 0
                \end{align*}
            \end{minipage}\\
            \vspace{\baselineskip} 
            Donde se hace uso de $<$ si el signo de la correlación muestral es negativa y, 
            en caso contrario, se usa $>$, y donde $\rho$ se refiere al coeficiente de correlación 
            que hay entre dos macronutrientes en una dieta.\\
            \vspace{\baselineskip} 
            % \begin{minipage}{0.5\textwidth}
                % \centering                
                % \begin{tabular}{lrr}
                    % \toprule
                    % \multicolumn{3}{c}{DASH}\\
                    % \midrule
                        % Macronutrientes & Valores P & Estadístico r \\
                    % \midrule
                        % Carbs Protein & 0.0000 & -0.6850 \\
                        % Carbs Fat     & 0.0000 & -0.7650 \\
                        % Protein Fat   & 0.0161 &  0.0550 \\
                    % \bottomrule
                % \end{tabular}
            % \end{minipage}%
            % \begin{minipage}{0.5\textwidth}
                % \centering                
                % \begin{tabular}{lrr}
                    % \toprule
                    % \multicolumn{3}{c}{Keto}\\
                    % \midrule
                        % Macronutrientes & Valores P & Estadístico r \\
                    % \midrule
                        % Carbs Protein & 0 & -0.4780 \\
                        % Carbs Fat     & 0 & -0.4621 \\
                        % Protein Fat   & 0 & -0.5578 \\
                    % \bottomrule
                % \end{tabular}
            % \end{minipage}\\
            % \vspace{\baselineskip} 
            % \begin{minipage}{0.5\textwidth}
                % \centering                
                % \begin{tabular}{lrr}
                    % \toprule
                    % \multicolumn{3}{c}{Mediterranean}\\
                    % \midrule
                        % Macronutrientes & Valores P & Estadístico r \\
                    % \midrule
                        % Carbs Protein & 0 & -0.66431 \\
                    %    Carbs Fat     & 0 & -0.65294 \\
                        % Protein Fat   & 0 & -0.13236 \\
                    % \bottomrule
                % \end{tabular}
            % \end{minipage}%
            % \begin{minipage}{0.5\textwidth}
                % \centering                
                % \begin{tabular}{lrr}
                    % \toprule
                    % \multicolumn{3}{c}{Paleo}\\
                    % \midrule
                        % Macronutrientes & Valores P & Estadístico r \\
                    % \midrule
                        % Carbs Protein & 0 & -0.6293 \\
                        % Carbs Fat     & 0 & -0.6284 \\
                        % Protein Fat   & 0 & -0.2089 \\
                    % \bottomrule
                % \end{tabular}
            % \end{minipage}\\
            % \vspace{\baselineskip} 
            % \begin{minipage}{\textwidth}
                % \centering
                % \begin{tabular}{lrr}
                    % \toprule
                    % \multicolumn{3}{c}{Vegana}\\
                    % \midrule
                        % Macronutrientes & Valores P & Estadístico r \\
                    % \midrule
                        % Carbs Protein & 0 & -0.3740 \\
                        % Carbs Fat     & 0 & -0.8668 \\
                        % Protein Fat   & 0 & -0.1381 \\
                    % \bottomrule
                % \end{tabular}
            % \end{minipage}\\ 
            % \vspace{\baselineskip} 
            Como todos los valores p son $0$ se tiene que existe una correlación y que los valores 
            reportados en \fullref{sec:biva} se vuelven buenas estimaciones de las correlaciones entre los 
            macronutrientes. Por lo que se puede construir un modelo lienal que sea confiable 
            sobre las tendencias de sus predicciones que realice, teniendo los siguientes parámetros por 
            cada dieta, donde $m$ es la pendiente y $b$ es la intercepción con el eje $y$:

            \begin{center}
                \includegraphics[width=0.9\textwidth]{Resources/Bivariado/RegressionDash.png}\\
                \includegraphics[width=0.9\textwidth]{Resources/Bivariado/RegressionKeto.png}\\
                \includegraphics[width=0.9\textwidth]{Resources/Bivariado/RegressionMediterranean.png}\\
                \includegraphics[width=0.9\textwidth]{Resources/Bivariado/RegressionPaleo.png}\\
                \includegraphics[width=0.9\textwidth]{Resources/Bivariado/RegressionVegan.png}
            \end{center}

            Donde se puede apreciar que tienen un buen ajuste según el valor absoluto de la 
            correlación que tengan, es decir, los macronutrientes de las recetas de una dieta 
            siguen la tendencia de la recta de regresión generada mostrando así la correlación 
            entre los macronutrientes. En específico, como cada dieta sigue ciertos patrones 
            sobre las composiciones de los macronutrientes de una receta.
        }
    }

    \newpage

    \addcontentsline{toc}{section}{Anexos}
    \appendix
    {
        \section{Marco Teórico}\label{anexo:A}
        {
        La dieta es uno de los principales factores de riesgo de las enfermedades 
        crónicas, y las enfermedades sensibles a la dieta contribuyen en gran medida 
        a los costes sanitarios mundiales. Se han propuesto literalmente miles de 
        \emph{dietas}, que pueden describirse en términos generales como basadas en 
        creencias, en alimentos específicos o en nutrientes; centradas en la 
        pérdida de peso o en el aumento de peso (muscular); dietas de desintoxicación 
        (detox) y dietas diseñadas por razones médicas específicas.\cite{marvastipopular} \\
        \vspace{\baselineskip}
        Las \emph{dietas de moda} son dietas populares durante un tiempo sin basarse 
        necesariamente en una recomendación dietética estándar. A menudo promueven 
        una pérdida de peso irracionalmente rápida o afirmaciones de salud sin 
        sentido, y se anuncian como dietas que requieren poco esfuerzo por parte de 
        quien las sigue. La promesa de ganancias fáciles, combinada con la presión 
        social para lograr un determinado tipo de cuerpo, puede dejar al público 
        susceptible a afirmaciones infundadas o exageradas.\cite{marvastipopular} \\
        \vspace{\baselineskip}
        Las dietas estudiadas desde una perspectiva estadística en el presente 
        trabajo, son englobadas en las \emph{dietas de moda}, que a veces son referidas 
        como \emph{dietas sin evidencia científica}. Siendo  la dieta DASH la única 
        que cuenta con algún tipo de fundamento.
        
        \subsection{DASH (Dietary Approaches to Stop Hypertension)}
        {
            \cite{marvastipopular} La dieta DASH (Enfoques Dietéticos para Detener la 
            Hipertensión) es un patrón dietético diseñado específicamente para ayudar 
            a reducir la presión arterial y promover la salud general del corazón. Hace 
            hincapié en el consumo de una variedad de alimentos ricos en nutrientes, 
            como frutas, verduras, cereales integrales, proteínas magras y productos 
            lácteos bajos en grasa, y en la limitación de la ingesta de sodio, grasas 
            saturadas y azúcares añadidos. 
        }
            
        \subsection{Dieta Keto}
        {
            \cite{marvastipopular} Una dieta baja en hidratos de carbono (baja en 
            carbohidratos) es un patrón alimentario que restringe la ingesta de 
            carbohidratos, sustituyéndolos normalmente por mayores cantidades de 
            proteínas y grasas. La dieta cetogénica es una forma de dieta baja en 
            carbohidratos con un alto contenido en grasas en relación con la ingesta 
            de proteínas y carbohidratos.\\
            \vspace{\baselineskip}
            El objetivo de la dieta cetogénica es inducir la cetosis, un estado 
            metabólico que se produce cuando el cuerpo quema grasa para obtener 
            energía en lugar de glucosa, lo que induce la pérdida de peso. 
        }
                
        \subsection{Dieta Mediterránea}
        {
            \cite{marvastipopular} La dieta mediterránea es un patrón alimentario 
            inspirado en los hábitos alimenticios tradicionales de los países 
            situados a orillas del mar Mediterráneo. Se caracteriza por un alto consumo 
            de frutas, verduras, cereales integrales, legumbres, frutos secos y 
            aceite de oliva; un consumo moderado de pescado y aves; y un bajo 
            consumo de carnes rojas, alimentos procesados y dulces.
        }
                    
        \subsection{Dieta Paleo (Paleolítica)}
        {
            \cite{marvastipopular} La dieta paleo, también conocida como dieta 
            paleolítica o dieta del hombre de las cavernas, es un enfoque 
            dietético que pretende imitar los hábitos alimentarios de nuestros 
            antiguos antepasados del Paleolítico. \\
            \vspace{\baselineskip}
            Hace hincapié en el consumo de alimentos integrales y no procesados 
            que habrían estado al alcance de los primeros humanos, como carnes magras, 
            pescado, frutas, verduras, frutos secos y semillas, y excluye los cereales, 
            las legumbres, los productos lácteos, los alimentos procesados y los 
            azúcares añadidos.
        }
        
        \subsection{Dieta Vegana}
        {
            \cite{marvastipopular} La dieta vegana es un patrón dietético basado en 
            plantas que excluye el consumo de todos los productos de origen animal. Se 
            centra en el consumo de una variedad de alimentos de origen vegetal, como 
            frutas, verduras, cereales legumbres, frutos secos y semillas.\\
            \vspace{\baselineskip}
            Es importante señalar que, aunque las dietas veganas pueden ser 
            nutricionalmente adecuadas, debe prestarse atención a garantizar una 
            ingesta suficiente de nutrientes esenciales como proteínas, hierro, 
            calcio, vitamina B12 y ácidos grasos omega-3.
        }
        }

        \newpage

        \section{Estratificación por Tipo de Dieta}\label{anexo:C}
        {
            \subsection{Dieta DASH}\label{anexo:C_dash}
            {
            Una receta de esta dieta tendrá que, en promedio, el $49\%$ de sus 
            macronutrientes son carbohidratos (provenientes de frutas, vegetales 
            y granos enteros); el $29\%$ son grasas que, por su naturaleza, son 
            saludables; y el $22\%$ son proteínas, las cuáles provienen de carnes margas.
            Aunque esta dieta se menciona ser saludable para la salud cardiovascular, 
            no implica que exista un balance o equilibrio en los macronutrientes 
            consumidos por receta.\newline
            
            Debido a la desviación estándar y rango intercuartilíco de las 
            proteínas y grasas, se tiene que estos macronutrientes se encuentran 
            concentrados en un rango más pequeño que el de los 
            carbohidratos. De lo mencionado, podría significar que la contribución 
            de los macronutrientes no son tan variadas como lo que se esperaría 
            contradiciendo que sea una dieta saludable, notando que es una dieta 
            rica en carbohidratos.

            \begin{center}
                \begin{tabular}{l|rrr}
                \toprule
                    Medida & Carbs & Protein & Fat \\
                \midrule
                    Media               & 0.491629 & 0.220058 & 0.288313 \\
                    $Q_1$               & 0.299070 & 0.094259 & 0.159870 \\
                    $Q_2$               & 0.502760 & 0.180192 & 0.267363 \\
                    $Q_3$               & 0.683241 & 0.311076 & 0.393803 \\
                    Desviación Estándar & 0.249919 & 0.161171 & 0.182336 \\
                    Mínimo              & 0.001526 & 0.000000 & 0.000000 \\
                    Máximo              & 1.000000 & 0.833467 & 0.973404 \\
                    Asimetría de Fisher & -0.009756 & 1.029402 & 0.766955 \\
                \bottomrule
                \end{tabular}\\
                \vspace{0.5cm}
                \includegraphics[width=0.9\textwidth]{Resources/EDA/Dash_1.png}
            \end{center}

            Al usar los tipos de cocina, se puede apreciar el como en cada una de 
            ellas tiene un comportamiento ligeramente diferente, es decir, al considerar 
            las cocinas en la dieta DASH se puede apreciar como los macronutrientes no 
            tienen una diferencia notoria sino que son cambios ligeros sobre su comportamiento. 
            Por ello, se puede decir que la dieta DASH es consistente sobre las cocinas y 
            que se comporta de la misma manera, aunque si se hace uso de un estadístico para 
            medir esta diferencia, tendrá un valor bajo más no sería nulo.

            \begin{center}
                \includegraphics[width=0.9\textwidth]{Resources/EDA/Dash_2.png}
            \end{center}
            }

            \subsection{Dieta Keto}\label{anexo:C_keto}
            {
            Una receta de esta dieta tendrá que, en promedio, el $50\%$ de 
            sus macronutrientes son grasas, esto se relaciona con el hecho de 
            que se intenta inducir la ketosis (principio en que se basa esta         
            dieta); el $30\%$ son proteínas, notando que se intenta reducir 
            el consumo de carbohidratos; y el $20\%$ son carbohidratos, 
            resaltando ser una dieta baja en carbohidratos.\newline 
            
            Como la proporciones de carbohidratos cuenta con un sesgo positivo, se 
            tiene que refuerza el hecho de ser una dieta baja en carbohidratos. 
            De los aportes de grasas, se observa que su sesgo es despreciable 
            implicando que existen recetas tanto con aportes altos de este 
            macronutriente mientras que hay recetas con 
            una contribución baja o nula del mismo.
        
            \begin{center}
                \begin{tabular}{l|lll}
                    \toprule
                        Medida & Carbs & Protein & Fat \\
                    \midrule
                        Media               & 0.198449 & 0.302718 & 0.498833 \\
                        $Q_1$               & 0.085336 & 0.159164 & 0.408251 \\
                        $Q_2$               & 0.156879 & 0.304214 & 0.506381 \\
                        $Q_3$               & 0.264915 & 0.410555 & 0.592312 \\
                        Desviación Estándar & 0.155979 & 0.166667 & 0.165075 \\
                        Mínimo              & 0.002060 & 0.000000 & 0.000000 \\
                        Máximo              & 1.000000 & 0.856868 & 0.997940 \\
                        Asimetría de Fisher & 1.562095 & 0.313745 & -0.120900 \\
                    \bottomrule
                \end{tabular}\\
                \vspace{0.5cm}
                \includegraphics[width=0.9\textwidth]{Resources/EDA/Keto_1.png}
            \end{center}

            Para la dieta keto, lo más importante es mantener un consumo alto de grasas, o 
            por al menos mayor que el de carbohidratos. Por lo que al ver la distribución de 
            las grasas contra los carbohidratos en cada una de las cocinas se verifica. 
            Además se puede ver como la tendencia es tener un consumo alto de grasas, aunque, por 
            ejemplo la cocina asiática, su consumo de grasas figura ser más bajo pero tienen 
            un sesgo negativo, por ello se puede determinar que sigue el principio de la dieto keto.

            \begin{center}
                \includegraphics[width=0.9\textwidth]{Resources/EDA/Keto_2.png}
            \end{center}
            }

            \subsection{Dieta Mediterránea}\label{anexo:C_medit}
            {
            Una receta de esta dieta tendrá que, en promedio, el $42\%$ de sus 
            macronutrientes son carbohidratos, esto debido a un alto consumo de 
            productos como, frutas, vegetales y granos enteros; el $30\%$ son 
            grasas, resaltando un alto consumo de nueces y aceite de oliva, como 
            también un consumo moderado de pescado; y el $28\%$ son proteínas, 
            vinculado con un consumo moderado de pescado y aves de corral, y un 
            bajo consumo de carnes rojas.\newline 
            
            Las proteínas y grasas tienen un alto 
            sesgo positivo junto con una desviación estándar bajo, esto representa 
            que muchas de las recetas tendrán bajas proporciones de estos macronutrientes.

            \begin{center}
                \begin{tabular}{l|lll}
                    \toprule
                        Medida & Carbs & Protein & Fat \\
                    \midrule
                        Media               & 0.422803 & 0.280190 & 0.297007 \\
                        $Q_1$               & 0.249507 & 0.159930 & 0.180738 \\
                        $Q_2$               & 0.438382 & 0.229058 & 0.268950 \\
                        $Q_3$               & 0.605733 & 0.377918 & 0.390790 \\
                        Desviación Estándar & 0.212640 & 0.162485 & 0.160349 \\
                        Mínimo              & 0.006733 & 0.005036 & 0.001731 \\
                        Máximo              & 0.992746 & 0.887557 & 0.968722 \\
                        Asimetría de Fisher & -0.123218 & 0.962969 & 0.878710 \\
                    \bottomrule
                \end{tabular}\\
                \vspace{0.5cm}
                \includegraphics[width=0.9\textwidth]{Resources/EDA/Mediterranean_1.png}
            \end{center}       

            Debido que para esta receta se cuenta con la región de dónde proviene, la 
            comparativa se vuelve respecto a la distribución de los macronutrientes en 
            la cocina del mediterráneo. Se tiene que las grasas se encuentran más variadas, 
            es decir, se puede ver que en 
            algunas cocinas toman valores más altos o más bajos en las grasas; al considerar 
            las proteínas se puede apreciar este mismo fenómeno pero en menor medida. Esto se relaciona 
            a que puede existir una diversificación de la dieta para adecuarse a los 
            productos y alimentos propios de una cierta región.

            \begin{center}
                \includegraphics[width=0.9\textwidth]{Resources/EDA/Mediterranean_2.png}
            \end{center}
            }

            \subsection{Dieta Paleo}\label{anexo:C_paleo}
            {
            Una receta de esta dieta tendrá que, en promedio, el $38\%$ de sus 
            macronutrientes son grasas y el $37\%$ son carbohidratos, esto se 
            relaciona con el consumo de productos como frutas, vegetales, nueces 
            y semillas; y el $25\%$ son proteínas cuyas principales fuentes son 
            carnes margas y pescado. La posible limitante de alimentos asociados 
            a proteínas	y grasas podría impactar en que las recetas estén hechas 
            con los mismos productos dentro de la misma región geográfica. Estos 
            se relacionaría con una baja variedad en la presencia de estos 
            macronutrientes.

            \begin{center} 
                \begin{xtabular}{w{l}{\widthof{Desviación Estándar}}|lll}
                    \toprule
                        Medida & Carbs & Protein & Fat \\
                    \midrule
                        Media               & 0.369912 & 0.250207 & 0.379881 \\
                        $Q_1$               & 0.191815 & 0.103933 & 0.257510 \\
                        $Q_2$               & 0.350726 & 0.207286 & 0.382872 \\
                        $Q_3$               & 0.512822 & 0.375863 & 0.488153 \\
                        Desviación Estándar & 0.219894 & 0.174915 & 0.174737 \\
                        Mínimo              & 0.003612 & 0.000000 & 0.006210 \\
                        Máximo              & 0.976322 & 0.858503 & 0.968835 \\
                        Asimetría de Fisher & 0.473617 & 0.710366 & 0.325286 \\
                    \bottomrule
                \end{xtabular}\\
                \vspace{0.5cm}
                \includegraphics[width=0.9\textwidth]{Resources/EDA/Paleo_1.png}
            \end{center}       

            \begin{center}
                \includegraphics[width=0.9\textwidth]{Resources/EDA/Paleo_2.png}
            \end{center}

            Que las proteínas y grasas tengan distribuciones diferentes entre 
            las cocinas se relaciona con el hecho de que cada región geográfica 
            tiene diferentes disponibilidad de recursos alimentarios, haciendo que 
            este fenómeno impacte en las recetas que se pueden hacer con lo que 
            esté disponible.
            }

            \subsection{Dieta Vegana}\label{anexo:C_vegan}
            {
            Una receta de esta dieta tendrá que, en promedio, el $60\%$ de 
            sus macronutrientes son carbohidratos, que provienen de fuentes 
            como vegetales, frutas, cereales y legumbres; el $25\%$ son grasas, 
            relacionadas con el consumo de nueces y semillas; y el $15\%$ son 
            proteínas, esto debido a un nulo consumo de alimentos de origen 
            animal y que estas fuentes son reemplazadas por fuentes vegetales.\newline

            En las proteínas, se puede observar un rango intercuartilico reducido y una 
            desviación estándar reducida, esto evoca a que las recetas tengan bajos 
            aportes de proteínas así como también los valores de aportes se concentren 
            en un rango reducido.\newline

            Se tiene que entre las cocinas se encuentran diferencias en las 
            proteínas y carbohidratos, representando que las distinciones sobre 
            las dietas se haya que tan favorecidas son las proteínas, es decir, 
            en qué cocinas se pueden encontrar alimentos con altos aportes de proteínas, 
            como lo sería la asiática que también es la que sus aportes de carbohidratos 
            tienden a ser bajos. 

            \begin{center}
                \begin{tabular}{l|lll}
                    \toprule
                        Medida & Carbs & Protein & Fat \\
                    \midrule
                        Media               & 0.593169 & 0.148801 & 0.258029 \\
                        $Q_1$               & 0.502866 & 0.085853 & 0.143034 \\
                        $Q_2$               & 0.625406 & 0.139807 & 0.231985 \\
                        $Q_3$               & 0.713660 & 0.190724 & 0.344991 \\
                        Desviación Estándar & 0.171149 & 0.086151 & 0.160262 \\
                        Mínimo              & 0.000330 & 0.001921 & 0.000112 \\
                        Máximo              & 0.986872 & 0.647416 & 0.994887 \\
                        Asimetría de Fisher & -0.735666 & 1.439623 & 1.092007 \\
                    \bottomrule
                \end{tabular}\\
                \vspace{0.5cm}
                \includegraphics[width=0.9\textwidth]{Resources/EDA/Vegan_1.png}
            \end{center}       

            \begin{center}
                \includegraphics[width=0.9\textwidth]{Resources/EDA/Vegan_2.png}
            \end{center}   
            }
        }

        

        \newpage

        \section{Figuras Adicionales}\label{anexo:B}
        {
            \subsection{Correlograma de los Macronutrientes En las Diferentes Dietas}\label{anexo:B.1}
            {
                \begin{center}
                   \includegraphics[height=0.9\textheight]{Resources/Bivariado/Correlation.png}
                \end{center}
            }
        }
    }
        
    \newpage
    
    {
        \printbibliography[heading=bibintoc,title={Referencias Bibliográficas}]
    }

\end{document}