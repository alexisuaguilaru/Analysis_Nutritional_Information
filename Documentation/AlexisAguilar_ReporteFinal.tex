\documentclass[12pt,a4paper]{article}

\usepackage{geometry}
\geometry{
    left=2cm, 
    right=2cm,
    top=3cm,  
    bottom=2cm
}

\usepackage[spanish]{babel}
\usepackage[utf8]{inputenc}
\usepackage{amsmath}

\usepackage{graphicx}
\usepackage{wrapfig}
\usepackage{makecell}

\usepackage{booktabs}
\usepackage{enumitem}
\usepackage{xcolor}


\usepackage{csquotes}
\usepackage{hyperref}
\usepackage[noabbrev,capitalize]{cleveref}
\usepackage[style=ieee]{biblatex}
\addbibresource{referencias.bib}

\newcommand{\fullref}[1]{%
  \hyperref[#1]{\cref*{#1}~\nameref*{#1}}%
}
\crefname{section}{Sección}{Secciones}       
\crefname{subsection}{Subsección}{Subsecciones}
\crefname{figure}{Figura}{Figuras}           
\crefname{table}{Tabla}{Tablas}

\usepackage{setspace}
\setstretch{1.25}
\setlength{\parindent}{0pt}

\begin{document}
    \begin{titlepage}
        \begin{minipage}[c]{0.1\textwidth}
            \includegraphics[width=\textwidth]{Resources/Cover/logo_unam.jpg}
        \end{minipage}
        \begin{minipage}{0.8\textwidth}
            \centering
            {\Large\textbf{Universidad Nacional Autónoma de México}\\}
            {\large\textbf{Escuela Nacional de Estudios Superiores\\\underline{Unidad Morelia}}}
        \end{minipage}
        \begin{minipage}[c]{0.1\textwidth}
            \includegraphics[width=\textwidth]{Resources/Cover/logo_enes.jpg}
        \end{minipage}
        \vspace{3cm}

        \centering
        {\large{Reporte Final\\}}
        {\Large\textbf{Análisis de Valores Nutricionales por Tipo de Dieta}}
        \vspace{2cm}

        {{PRESENTA:\\}}
        {\large\textbf{Alexis Uriel Aguilar Uribe}}
        \vspace{1cm} 

        {{PROFESORES:\\}}
        {\large\textbf{Dra.\ María Del Río Francos}}\\
        {\large\textbf{Dr.\ César Andrés Torres Miranda}}
        \vspace{2cm}

        {{GRADO\\}}
        {\large\textbf{Licenciatura en Tecnologías para la Información en Ciencias}}
        \vspace{2cm}

        \flushleft{
        {\textbf{Asignatura:\ }Estadística Descriptiva e Inferencial}
        \vspace{2cm}}

        \flushright{
        {\textbf{A:\ }\underline{26 de Mayo del 2025}}}
        \vfill
    \end{titlepage}

    \newpage

    \tableofcontents

    \newpage

    \section{Introducción}
    {
        Este trabajo tiene como fin de exponer el proceso llevado a cabo para 
        realizar el análisis estadístico de los valores nutricionales (macronutrientes) 
        que aportan las dietas: \emph{DASH} (Dietary Approaches to Stop Hypertension), 
        \emph{keto}, \emph{mediterránea}, \emph{paleo} (paleolítica) y \emph{vegana}.\\

        Siendo el principal enfoque el responder si hay una diferencia nutricional 
        significativa entre las diferentes dietas. En decir, hacer uso de 
        técnicas de estadística descriptiva e inferencial para probar si existe 
        una diferencia en los aportes nutricionales entre las distintas dietas que 
        están siendo estudiadas. La anterior prueba se basa en recetas de diferentes 
        cocinas a nivel mundial, y sobre éstas últimas serán auxiliares para realizar 
        un estudio más granulado sobre el comportamiento de las dietas en escenarios 
        más específicos.
    }

    \section{Objetivos Generales}
    {
        Para la realización de lo anterior expuesto, se puntualizan los objetivos del 
        proyecto:
        \begin{itemize}
            \item Realizar de un análisis estadístico de los macronutrientes en las 
            diferentes dietas con el fin de caracterizar sus aportes nutricionales y 
            sus distinciones en las diferentes cocinas.
            
            \item Conjeturar y probar hipótesis relacionadas a preguntas de interés 
            sobre los aportes nutricionales en cada dieta en base al análisis estadístico. 
            
            \item Probar si existe una diferencia significativa en los aportes 
            nutricionales entre las diferentes dietas con el fin de probar si cada 
            dieta de estudio es única.
        \end{itemize}
    }

    \newpage

    \section{Marco Teórico}\label{sec:theory}
    {
        La dieta es uno de los principales factores de riesgo de las enfermedades 
        crónicas, y las enfermedades sensibles a la dieta contribuyen en gran medida 
        a los costes sanitarios mundiales. Se han propuesto literalmente miles de 
        \emph{dietas}, que pueden describirse en términos generales como basadas en 
        creencias, en alimentos específicos o en nutrientes; centradas en la 
        pérdida de peso o en el aumento de peso (muscular); dietas de desintoxicación 
        (detox) y dietas diseñadas por razones médicas específicas.\cite{marvastipopular} \\
        
        Las \emph{dietas de moda} son dietas populares durante un tiempo sin basarse 
        necesariamente en una recomendación dietética estándar. A menudo promueven 
        una pérdida de peso irracionalmente rápida o afirmaciones de salud sin 
        sentido, y se anuncian como dietas que requieren poco esfuerzo por parte de 
        quien las sigue. La promesa de ganancias fáciles, combinada con la presión 
        social para lograr un determinado tipo de cuerpo, puede dejar al público 
        susceptible a afirmaciones infundadas o exageradas.\cite{marvastipopular} \\
        
        Las dietas estudiadas desde una perspectiva estadística en el presente 
        trabajo, son englobadas en las \emph{dietas de moda}, que a veces son referidas 
        como \emph{dietas sin evidencia científica}. Siendo  la dieta DASH la única 
        que cuenta con algún tipo de fundamento.
        
        \subsection{DASH (Dietary Approaches to Stop Hypertension)}
        {
            \cite{marvastipopular} La dieta DASH (Enfoques Dietéticos para Detener la 
            Hipertensión) es un patrón dietético diseñado específicamente para ayudar 
            a reducir la presión arterial y promover la salud general del corazón. Hace 
            hincapié en el consumo de una variedad de alimentos ricos en nutrientes, 
            como frutas, verduras, cereales integrales, proteínas magras y productos 
            lácteos bajos en grasa, y en la limitación de la ingesta de sodio, grasas 
            saturadas y azúcares añadidos. 
        }
            
        \subsection{Dieta Keto}
        {
            \cite{marvastipopular} Una dieta baja en hidratos de carbono (baja en 
            carbohidratos) es un patrón alimentario que restringe la ingesta de 
            carbohidratos, sustituyéndolos normalmente por mayores cantidades de 
            proteínas y grasas. La dieta cetogénica es una forma de dieta baja en 
            carbohidratos con un alto contenido en grasas en relación con la ingesta 
            de proteínas y carbohidratos.\\
            
            El objetivo de la dieta cetogénica es inducir la cetosis, un estado 
            metabólico que se produce cuando el cuerpo quema grasa para obtener 
            energía en lugar de glucosa, lo que induce la pérdida de peso.
        }
                
        \subsection{Dieta Mediterránea}
        {
            \cite{marvastipopular} La dieta mediterránea es un patrón alimentario 
            inspirado en los hábitos alimenticios tradicionales de los países 
            situados a orillas del mar Mediterráneo. Se caracteriza por un alto consumo 
            de frutas, verduras, cereales integrales, legumbres, frutos secos y 
            aceite de oliva; un consumo moderado de pescado y aves; y un bajo 
            consumo de carnes rojas, alimentos procesados y dulces.
        }
                    
        \subsection{Dieta Paleo (Paleolítica)}
        {
            \cite{marvastipopular} La dieta paleo, también conocida como dieta 
            paleolítica o dieta del hombre de las cavernas, es un enfoque 
            dietético que pretende imitar los hábitos alimentarios de nuestros 
            antiguos antepasados del Paleolítico. \\
            
            Hace hincapié en el consumo de alimentos integrales y no procesados 
            que habrían estado al alcance de los primeros humanos, como carnes magras, 
            pescado, frutas, verduras, frutos secos y semillas, y excluye los cereales, 
            las legumbres, los productos lácteos, los alimentos procesados y los 
            azúcares añadidos.
        }
        
        \subsection{Dieta Vegana}
        {
            \cite{marvastipopular} La dieta vegana es un patrón dietético basado en 
            plantas que excluye el consumo de todos los productos de origen animal. Se 
            centra en el consumo de una variedad de alimentos de origen vegetal, como 
            frutas, verduras, cereales legumbres, frutos secos y semillas.\\

            Es importante señalar que, aunque las dietas veganas pueden ser 
            nutricionalmente adecuadas, debe prestarse atención a garantizar una 
            ingesta suficiente de nutrientes esenciales como proteínas, hierro, 
            calcio, vitamina B12 y ácidos grasos omega-3.
        }

    }
    
    \newpage

    \section{Presentación de los Datos}
    {
        \subsection{Fuente de Datos}
        {
            El conjunto de datos con el que se está trabajando para este proyecto 
            se encuentran en \cite{dataset_macronutrients}, publicado por la comunidad 
            de Kaggle. Los datos consisten de un conjunto de recetas de diferentes 
            dietas y cocinas, además incluye información de los macronutrientes que 
            aporta cada receta.\\
            
            \cite{dataset_macronutrients} Aunque en la descripción ni en los metadatos del conjunto de datos se 
            haga mención de las fuentes explícitas de los datos ni el objetivo de 
            esta extracción, sí cuenta con una sección de cómo usar el conjunto de 
            datos, ideas de investigación y reconocimientos.\\
            
            De los apartados de cómo usar el conjunto de datos e ideas de investigación, 
            se encuentra una idea, implícita, de la información que se quería estudiar. 
            La principal información de interés se vuelve que es: el crear planes 
            alimenticios saludables, ya sea usando las recetas proporcionadas o creando 
            unas nuevas basadas en una dieta y cocina, y el estudiar la relación entre 
            dieta y salud.\\
            
            Del apartado de reconocimientos, se concluye que las recetas fueron 
            proporcionadas por diferentes creadores de las mismas y demás contribuidores 
            al conjunto de datos. 
        }

        \subsection{Interés del Estudio}
        {
            Se consultó \cite{marvastipopular} en sus 
            capítulos 4 y 8, de donde se proporciona un mejor entendimiento de la 
            importancia de los macronutrientes y una descripción general de las 
            dietas en este trabajo, resultando interesante que en cada dieta se 
            consumen diferentes alimentos y productos con ciertas características 
            para ya sea respetar alguna creencia, fundamento o cuota de macronutrientes. 
            De esto último, proporciona un indicio de que existe una diferencia entre 
            las dietas a nivel de sus aportes nutricionales, por lo tanto, lo que se 
            quiere realizar es probar esta diferencia de manera significativa haciendo 
            uso de la estadística y, en caso de que la haya, mostrar que tanta es ésta 
            diferencia y sus implicaciones.
        }

        \subsection{Variables del Conjunto de Datos}
        {
            El conjunto de datos consta de las siguientes variables. Se menciona su 
            nombre, el tipo de variable y sus valores (en total y únicos):
            
            \begin{center}
                \begin{tabular}{r|llrr}
                    \toprule
                    Variable & Nombre & Tipo & Cantidad de Datos & Valores Únicos\\
                    \midrule
                    1 & Diet\_type & Cualitativa Nominal & 7806 & 5 \\
                    2 & Recipe\_name & Cualitativa Nominal & 7806 & 7062\\
                    3 & Cuisine\_type & Cualitativa Nominal & 7806 & 19\\
                    4 & Protein & Cuantitativa Continua & 7806 & 6060\\
                    5 & Carbs & Cuantitativa Continua & 7806 & 6618\\
                    6 & Fat & Cuantitativa Continua & 7806 & 6322\\
                    \bottomrule
                \end{tabular}
            \end{center}
            
            La variable \emph{Recipe\_Name} no es relevante para este trabajo pero figura 
            dentro del dataset. Se hace mención que el conjunto de datos no presenta 
            valores faltantes.
        }

        \subsection{Ejemplo de Registros en el Conjunto de Datos}\label{subsec:ejemplos}
        {
            Para ejemplificar como luce el conjunto de datos, se presente 
            una instancia de cada tipo de dieta:
            
            \begin{center}       
                \begin{tabular}{lll}
                    \toprule
                    \textbf{Diet\_type} & \textbf{Recipe\_name} & \textbf{Cuisine\_type} \\
                    \midrule
                    dash          & Old Fashioned                     & world \\
                    keto          & Keto Egg Drop Soup                & chinese \\
                    mediterranean & Mediterranean Mix                 & mediterranean \\
                    paleo         & Easy Paleo Herb Gravy recipes     & french \\
                    vegan         & Braised Green Beans with Tomatoes & mediterranean \\
                    \bottomrule
                \end{tabular}
            \end{center}
            
            \begin{center}
                \begin{tabular}{rrr}
                    \toprule
                    \textbf{Protein} & \textbf{Carbs} & \textbf{Fat} \\
                    \midrule
                    0.12 &  9.66 &  0.02 \\
                    21.31 &  9.11 & 60.88 \\
                    8.11 &  9.59 & 14.64 \\
                    23.56 & 39.05 & 42.25 \\
                    17.49 & 77.86 & 70.20 \\
                    \bottomrule
                \end{tabular}
            \end{center}
            }
    }

    \newpage

    \section{Estadística Descriptiva}\label{sec:eda}
    {
        \subsection{Preprocesamiento (Transformación) de los Datos}
        {
            De las instancias presentadas en \fullref{subsec:ejemplos}, se tiene 
            que los valores de los macronutrientes pueden tomar un amplio rango 
            de valores, esto puede generar un conflicto al momento de generar una 
            comparativa entre dietas, debido a principalmente los rangos de valores. 
            Para resolver esta situación los valores de los macronutrientes son 
            normalizados con la norma $l1$, es decir, se calcula el total de macronutrientes 
            (que se guarda como otra variable en \emph{Total\_macronutrients}) de cada 
            receta y cada macronutriente se divide por este total.\\

            \begin{center}
                \begin{tabular}{l|rrrrr|r}
                \toprule
                    & dash & keto & mediterranean & paleo & vegan & Suma Dietas \\
                \midrule
                    kosher           & 5    & 0    & 0    & 2    & 0    & 7    \\
                    caribbean        & 3    & 7    & 1    & 6    & 1    & 18   \\
                    central europe   & 9    & 11   & 1    & 9    & 4    & 34   \\
                    japanese         & 9    & 10   & 2    & 5    & 24   & 50   \\
                    eastern europe   & 10   & 11   & 3    & 27   & 4    & 55   \\
                    middle eastern   & 21   & 17   & 26   & 12   & 15   & 91   \\
                    indian           & 20   & 12   & 3    & 9    & 48   & 92   \\
                    chinese          & 38   & 38   & 1    & 26   & 17   & 120  \\
                    asian            & 24   & 11   & 12   & 12   & 67   & 126  \\
                    south american   & 54   & 21   & 10   & 21   & 31   & 137  \\
                    south east asian & 31   & 34   & 8    & 29   & 46   & 148  \\
                    nordic           & 32   & 35   & 31   & 45   & 9    & 152  \\
                    mexican          & 61   & 60   & 17   & 48   & 38   & 224  \\
                    british          & 64   & 90   & 4    & 54   & 27   & 239  \\
                    world            & 234  & 6    & 6    & 3    & 10   & 259  \\
                    french           & 150  & 163  & 61   & 154  & 76   & 604  \\
                    italian          & 165  & 234  & 148  & 171  & 81   & 799  \\
                    mediterranean    & 176  & 89   & 1274 & 106  & 99   & 1744 \\
                    american         & 639  & 663  & 145  & 535  & 925  & 2097 \\
                \midrule
                    Suma Cocinas     & 1745 & 1512 & 1753 & 1274 & 1522 & 7806 \\
                \bottomrule
                \end{tabular}
            \end{center}

            Al considerar la cantidad de recetas que hay por dieta y cocina, se 
            descubre que, para ciertas grupos o configuraciones no contienen recetas, 
            por lo que para mitigar esta falta de instancias lo que se realizar es juntar 
            los \emph{Cuisine\_type} en base a la cercanía geográfica, esto debido a 
            que son colindantes comparten historia, cultura y, lo más relevante, ideas 
            gastronómicas. Por ello, las \emph{Cuisine\_type} se reagrupan de la siguiente 
            manera: 

            \begin{center}
                \begin{tabular}{l|l}
                \toprule
                    Grupo de \emph{Cuisine\_type} & \emph{Cuisine\_type} \\
                \midrule
                    american & american \\
                    mediterranean & mediterranean \\
                    world & world \\
                    latin american and caribbean & mexican, south american, caribbean \\
                    european & \makecell{italian, french, nordic, eastern europe,\\central europe, kosher, british} \\
                    asian & \makecell{chinese, indian, south east asian,\\middle eastern, asian, japanese} \\
                \bottomrule
                \end{tabular}
            \end{center}

        }
    }

    \newpage

    \section{Análisis Bivariado}\label{sec:biva}
    {

    }

    \newpage

    \section{Muestreo e Intervalos de Confianza}
    {

    }

    \newpage

    \section{Pruebas de Hipótesis}
    {

    }

\end{document}