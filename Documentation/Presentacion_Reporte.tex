\documentclass{beamer}
\usepackage[english,spanish]{babel}
\usetheme{Boadilla}

\usepackage{graphicx}
\usepackage{xcolor}

\usepackage{csquotes}
\usepackage{hyperref}
\usepackage[style=ieee]{biblatex}
\addbibresource{referencias.bib}

\title[EDA de Valores Nutricionales por Dieta]{Análisis Estadístico de Valores Nutricionales por Tipo de Dieta}

\setbeamercolor{footline}{fg=black, bg=black!25}
\setbeamertemplate{footline}{
    \begin{beamercolorbox}[wd=\paperwidth, ht=2mm, dp=2mm, center]{footline}
        \begin{columns}[T]
            \begin{column}{0.5\textwidth}
                \centering
                \insertshorttitle
            \end{column}
            \begin{column}{0.5\textwidth}
                \centering
                \insertsection
            \end{column}
        \end{columns}
    \end{beamercolorbox}
}
\setbeamertemplate{navigation symbols}{}

\usepackage{hyperref}
\hypersetup{
    colorlinks=true,
    urlcolor=blue,
    linkcolor=black,
    citecolor=blue
}

\begin{document}

    \begin{frame}[plain]
        \begin{columns}[T]
            \begin{column}{0.08\textwidth}
                \centering
                \includegraphics[width=\textwidth]{Resources/logo_unam.jpg}
            \end{column}
            \begin{column}{0.60\textwidth}
                \centering
                {\small Universidad Nacional Autónoma de México\par
                Escuela Nacional de Estudios Superiores\par
                Morelia}
            \end{column}
            \begin{column}{0.08\textwidth}
                \centering
                \includegraphics[width=\textwidth]{Resources/logo_enes.jpg}
            \end{column}
        \end{columns}
        \vspace{1.5cm}
        \begin{center}
            {\bfseries Reporte}\par
            {\Large Análisis Estadístico de Valores Nutricionales}\par
            {\Large por Tipo de Dieta}\par
            \vspace{0.5cm}
            {\bfseries Alexis Aguilar}
        \end{center}
        \vspace{1.5cm}
        \begin{columns}[T]
            \begin{column}{0.5\textwidth}
                \raggedright
                {\footnotesize Estadística Descriptiva e Inferencial}
            \end{column}
            \begin{column}{0.5\textwidth}
                \raggedleft
                {\footnotesize A: \underline{7 de Abril del 2025}}
            \end{column}
        \end{columns}
    \end{frame}

    \begin{frame}[plain]{Índice}
        \tableofcontents
    \end{frame}

    \section{Presentación de los Datos}

    \begin{frame}{Fuente de Datos}
        \begin{itemize}
            \item<1->El conjunto de datos consta de recetas 
            de diferentes dietas y cocinas, incluye también 
            la información de los macronutrientes (carbohidratos, 
            proteínas, lípidos) que aportan cada receta. Se 
            encuentra publicado en \cite{dataset_macronutrients}
            \item<2->El crear planes alimenticios saludables, ya 
            sea usando las recetas proporcionadas o creando unas 
            nuevas basadas en una dieta y cocina, y el estudiar 
            la relación entre dieta y salud
            \item<3->Las recetas fueron proporcionadas por 
            diferentes creadores de las mismas y demás 
            contribuidores al conjunto de datos
        \end{itemize}
    \end{frame}

    \begin{frame}{Interés del Estudio}
        \begin{itemize}
            \item<1->\cite{marvastipopular}Como en cada dieta se consumen diferentes 
            alimentos y productos con ciertas características 
            para ya sea respetar alguna creencia, fundamento o 
            cuota de macronutrientes
            \item<2->Por lo que el interés del trabajo 
            es el probar si existe una diferencia o distinción 
            entre las dietas en base a sus aportes de macronutrientes
        \end{itemize}
    \end{frame}

    \begin{frame}{Variables del Conjunto de Datos}
        El conjunto de datos con el se trabajará consta de 
        las siguientes variables:
        \begin{itemize}
            \item<2->\textbf{Diet\_type}: [Nominal] Representa 
            el tipo de dieta (DASH, keto, mediterránea, paleo, vegana) 
            a la que pertenece una receta. Estratificar las recetas 
            y estudiarlas de una manera más granular
            \item<3->\emph{Recipe\_name}: [Nominal] Nombre de 
            la receta. No es una variable relevante para el trabajo
            \item<4->\textbf{Cuisine\_type}: [Nominal] Representa 
            a qué (estilo de) cocina o región (mexicana, americana, 
            italiana, entre otras) pertenece una receta
        \end{itemize}
    \end{frame}

    \begin{frame}{Variables del Conjunto de Datos}
        Otras variables que serán eje para el trabajo son:
        \begin{itemize}
            \item<1->\textbf{Protein(g)}: [Continua] Representa la 
            cantidad de proteínas en gramas contenidas en una receta
            \item<2->\textbf{Carbs(g)}: [Continua] Representa la 
            cantidad de carbohidratos en gramas contenidas en una receta
            \item<3->\textbf{Fat(g)}: [Continua] Representa la 
            cantidad de grasas en gramas contenidas en una receta
        \end{itemize}
    \end{frame}

    \section{Referencias Bibliográficas}

    \begin{frame}{Referencias}
        \printbibliography
    \end{frame}
\end{document}