\documentclass[12pt,a4paper]{article}

\usepackage{geometry}
\geometry{
    left=2cm, 
    right=2cm,
    top=3cm,  
    bottom=2cm
}

\usepackage[spanish,english]{babel}
\usepackage[utf8]{inputenc}
\usepackage{amsmath}

\usepackage{graphicx}
\usepackage{wrapfig}

\usepackage{booktabs}
\usepackage{enumitem}
\usepackage{xcolor}

\usepackage{setspace}
\setstretch{1.5}
\setlength{\parindent}{0pt}

\usepackage{titlesec}
\titleformat*{\section}{\large\bfseries}
\titleformat*{\subsection}{\bfseries}

\begin{document}
    
    \begin{center}
        {\Large
            \textbf{ Prueba de Hipótesis para Proyecto de Estadística Análisis de Valores Nutricionales por Tipo de Dieta}
        }
    \end{center}

    \section{Introducción}
    {
        Lo que haré es una prueba de hipótesis para cada tipo de dieta y otra 
        que englobe a las cinco dietas (que sería para probar mi objetivo principal 
        en este trabajo). Para ello, primero doy la versión general de lo que se quiere 
        hacer, luego el procedimiento previo a verificar el hipótesis y por último de 
        juego de hipótesis junto con la prueba de hipótesis que usaré. \\

        Para las diferentes pruebas siempre se maneja el mismo nivel de significancia 
        del $\alpha = 5\%$. Y la distribución de los macronutrientes, sin importar la 
        dieta, siguen una distribución beta (su soporte es $[0,1]$, los valores pertenecen 
        a ese intervalo y su forma no es parecida a una distribución uniforme).
    }

    \newpage

    \section{Dieta DASH}
    {  
        \subsection{Visión General}
        {
            Se quiere probar si la dieta está en balance nutricional, es decir, 
            si se llega a consumir la misma proporción de macronutrientes 
            de forma diaria.
        }

        \subsection{Procedimiento}
        {
            La cantidad diarias de comidas que se deberían que consumir son cinco, 
            esto es equivalente a consumir cinco recetas. Por lo que para probar 
            el balance bastaría con muestrear 50 días consumiendo comidas de la 
            dieta DASH, es decir, muestrear de forma aleatoria 250 recetas y agruparlas 
            en cinco en cinco recetas y obtener la proporción de macronutrientes que 
            se consumen con las cinco recetas. Esto último es el conjunto de datos 
            que se usará para la prueba de hipótesis.
        }

        \subsection{Prueba de Hipótesis}
        {
            Se tiene un conjunto de hipótesis por cada macronutriente, que lucen 
            de la forma:

            \begin{align*}
                H_0 :&\ \overline{x}_{M} = \frac{1}{3} \\ 
                H_1 :&\ \overline{x}_{M} \ne \frac{1}{3}
            \end{align*}

            Donde $M$ se refiere a uno de los tres macronutrientes con los que 
            se cuenta. El signo de $\ne$ indica que la prueba será de dos colas 
            debido a que se quiere capturar cualquier diferencia significativa 
            por muy mínimo que sea.\\

            Como el tamaño de la muestra es lo suficientemente grande, se puede 
            suponer que la distribución de la media muestral es normal, por lo 
            tanto la prueba que se va usar es la Prueba t según la homocedasticidad y se esperaría que 
            diese resultados robustos.
        }

    }


    \section{Dieta Keto}
    {
        \subsection{Visión General}
        {
            Se quiere probar si en todas las cocinas se respeta el principio 
            de la dieta keto: El consumo de carbohidratos es menor que el 
            consumo de grasas
        }

        \subsection{Procedimiento}
        {
            Lo único que se tiene que hacer es el separar las recetas de la 
            dieta keto según el tipo de cocina y recuperar los datos acerca 
            de los carbohidratos y grasas.
        }

        \subsection{Prueba de Hipótesis}
        {
            Se tiene un conjunto de hipótesis por cada tipo de cocina, que 
            tienen la forma de:
            
            \begin{align*}
                H_0 :&\ \overline{c}_{C} = \overline{f}_{C} \\ 
                H_1 :&\ \overline{c}_{C} < \overline{f}_{C}
            \end{align*}

            Donde $\overline{c}$ y $\overline{f}$ indican las medias muestrales de carbohidratos y grasas
            reportadas por las recetas pertenecientes a la cocina $C$. El signo 
            $<$ indica que la prueba será de una cola hacia la derecha debido a 
            que se quiere probar si los carbohidratos son significativamente menores 
            que las grasas.\\

            Debido a que se cuenta con cocinas cuyo número de recetas son menores a 
            $30$ implica el uso inmediato de pruebas no paramétricas, es decir, usar 
            la Prueba de Rangos de Wilcoxon. Para el caso cuando se cuenta con los 
            suficientes datos, se usa la Prueba t dependiendo de la homocedasticidad  
            de los datos.
        }
    }

    \newpage


    \section{Dieta Mediterránea}
    {
        \subsection{Visión Genera}
        {
            Se quiere probar si existe una diferencia entre las 
            recetas que provienen del mediterraneo en contraste 
            con las recetas de los demás tipos de cocina.
        }

        \subsection{Procedimiento}
        {
            Se tiene primero que separar las recetas según si su 
            tipo de cocina es del mediterraneo o no, y recuperar los 
            valores referentes a los macronutrientes.
        }

        \subsection{Prueba de hipótesis}
        {
            Se tiene un conjunto de hipótesis por cada macronutriente, que 
            lucen de la forma:

            \begin{align*}
                H_0 :&\ F^{M}_{m} = F^{O}_{m} \\ 
                H_1 :&\ F^{M}_{m} \ne F^{O}_{m}
            \end{align*}

            Donde $F^{M}_{m}$ y $F^{O}_{m}$ se refieren a la distribución del 
            macronutriente $m$ en las recetas del mediterraneo y en otras cocinas, 
            respectivamente. El $\ne$ indica que será una prueba de dos colas debido 
            a que se quiere capturar cualquier mínima diferencia.\\

            Por como se definió, se tiene una prueba no paramétrica donde las distribuciones 
            de los datos son no normales, por lo que la prueba que se usa es la Prueba 
            Kolmogorov-Smirnov (aunado a que se cuentan con muestras de tamaño considerable).
        }
    }

    \newpage


    \section{Dieta Paleo}
    {
        \subsection{Visión Genera}
        {
            Se quiere probar si existe una diferencia significativa en los 
            aportes de proteínas en las diferentes cocinas.
        }

        \subsection{Procedimiento}
        {
            No es necesario realizar un procesamiento adicional a obtener las 
            recetas de cada tipo de cocina y de recuperar las proteínas	de ellas. Lo 
            que si se realiza es un filtrado para eliminar los tipos de cocinas 
            que tengan menos de 5 recetas, esto debido a la prueba que se usa.
        }

        \subsection{Prueba de hipótesis}
        {
            Debido a que las distribuciones son no normales y se podría suponer que 
            los tipo de recetas no están relacionadas, se tiene que lo que se va a probar 
            es que las medianas de los tipo de cocina son iguales. Por lo que el juego de hipótesis 
            se puede poner como:
        
            \begin{align*}
                H_0 :&\ \text{Todas las medianas son iguales} \\ 
                H_1 :&\ \text{Por al menos una mediana es diferente}
            \end{align*}

            Como lo anterior luce como una versión no paramétricas de ANOVAS e investigando 
            sobre pruebas para este estilo, la prueba que se usa es la Prueba de Kruskal-Wallis, 
            que permite usar datos de cualquier tipo de distribución, junto con la Prueba de Dunn 
            para determinar en qué parejas existe una diferencia significativa.
        }
    }

    \newpage


    \section{Dieta Vegana}
    {
        \subsection{Visión Genera}
        {
            Se quiere probar si en todas las cocinas se verifica que la ingesta de 
            carbohidratos es mayor que la de proteínas.
        }

        \subsection{Procedimiento}
        {
            Además de obtener las recetas por tipo de cocina y de recuperar los 
            valores de carbohidratos y proteínas, también es necesario de eliminar 
            los tipo de cocina donde se cuenta con registros (recetas) menos de 
            cinco recetas.
        }

        \subsection{Prueba de hipótesis}
        {
            Para tipo de cocina tiene un conjunto de hipótesis que tienen la forma:

            \begin{align*}
                H_0 :&\ \overline{p}_{C} = \overline{c}_{C} \\ 
                H_1 :&\ \overline{p}_{C} < \overline{c}_{C}
            \end{align*}

            Donde $\overline{p}$ y $\overline{c}$ indican las medias muestrales de proteínas y carbohidratos
            reportadas por las recetas pertenecientes a la cocina $C$. El signo 
            $<$ indica que la prueba será de una cola hacia la derecha debido a 
            que se quiere probar si las proteínas son significativamente menores 
            que los carbohidratos.\\

            Debido a que se cuenta con cocinas cuyo número de recetas son menores a 
            $30$ implica el uso inmediato de pruebas no paramétricas, es decir, usar 
            la Prueba de Rangos de Wilcoxon. Para el caso cuando se cuenta con los 
            suficientes datos, se usa la Prueba t dependiendo de la homocedasticidad  
            de los datos.
        }
    }

    \newpage


    \section{Entre todas las Dietas}
    {
        \subsection{Visión Genera}
        {
            Se quiere probar si existe una diferencia significativa en 
            los aportes nutricionales entre las diferentes dietas sin 
            importar el origen de las recetas
        }

        \subsection{Procedimiento}
        {
            No es necesario hacer nada más que recuperar las recetas junto 
            sus macronutrientes de cada dieta.
        }

        \subsection{Prueba de hipótesis}
        {
            Para cada tipo de dieta y macronutriente, lo que se prueba es 
            lo siguiente: Si sus distribuciones son significativamente 
            diferentes, en el sentido de que siguen distribuciones con diferentes 
            parámetros, se usa la Prueba de Kolmogorov-Smirnov debido a que se quiere 
            capturar cualquier mínima diferencia en las distribuciones. Se tiene el 
            siguiente conjunto de hipótesis:

            \begin{align*}
                H_0 :&\ F_m(x) = G_m(x) \\ 
                H_1 :&\ F_m(x) \ne G_m(x)
            \end{align*}

            Donde $F_m$ y $G_m$ es la distribución del macronutriente $m$ en dos 
            dietas diferentes. De $\ne$ indica usar una Prueba de Kolmogorov-Smirnov 
            de dos colas para detectar cualquier diferencia en las distribuciones.\\

            Para poder dar un valor para saber qué tan diferentes son las dietas, lo que se 
            hace es promediar los resultados de cada prueba donde $0$ significa que no existe 
            una diferencia significa y un $1$ que si la hay. Por lo tanto, mientras más se 
            aproxima a $1$, menos parecidas son las dietas.

        }
    }

    \newpage


\end{document}